\chapter{Envirronement} % (fold)
\label{cha:Envirronement}

Je réalise mon stage au sein de l'équipe Cortex au \textbf{L}aboratoire L\textbf{o}rrain de
\textbf{R}echerche en \textbf{I}nformatique et ses \textbf{A}pplications (\textbf{LORIA}).
Ce centre de recherche est partagé entre plusieurs instituts :
\begin{itemize}
   \item l'\textbf{INRIA}, \textbf{I}nstitut \textbf{N}ational de \textbf{R}echerche en \textbf{I}nformatique
      et en \textbf{A}utomatique (dont je suis dépendant)
   \item le \textbf{CNRS}, \textbf{C}entre \textbf{N}ational de \textbf{R}echerche \textbf{S}cientifique
   \item l'\textbf{INPL}, \textbf{I}nstitut \textbf{N}ational \textbf{P}olytechnique de \textbf{L}orraine
   \item l'\textbf{UHP}, \textbf{U}niversité \textbf{H}enri \textbf{P}oincaré, Nancy 1
   \item \textbf{Nancy 2}, Université Nancy 2
\end{itemize}

\section{L'INRIA} % (fold)
\label{sec:L'INRIA}

L'INRIA est un établissement de recherche publique à caractère scientifique et technologique. 8 centres de
recherches autonomes sont dispersés dans toute la France (Rocquencourt, Rennes, Sophia Antipolis, Grenoble,
Nancy, Bordeaux, Lille et Saclay).

L'INRIA s'organise autours de 174 équipes-projets à durée de vie limitée (12 ans maximums) et est composée
de 3350 scientifiques\footnote{Source : \url{http://www.inria.fr/institut/inria-en-bref/chiffres-cles}.}.

% section L'INRIA (end)

\section{LORIA} % (fold)
\label{sec:LORIA}

Comme précisé précédemment le LORIA est partagé entre différents instituts de recherche. Tout personnel confondu
ce laboratoire compte 450 personnes incluant 150 chercheurs et 1/3 de doctorants
\footnote{Source : \url{http://www.loria.fr/presentation-en}.}.

Les missions du LORIA sont :
\begin{itemize}
   \item \textbf{Recherche} fondamentale et appliquée au niveau international dans le domaine des Sciences et
      Technologies de l'Information et de la Communication.
   \item \textbf{Formation par la recherche} en partenariat avec les Universités lorraines
   \item \textbf{Transfert technologique} par le biais de partenariats industriels et par l'aide à la création
      d'entreprises
\end{itemize}

% section LORIA (end)

\section{L'équipe Cortex} % (fold)
\label{sec:Cortex}

L'équipe-projet Cortex est affiliée à la fois à l'INRIA, au CNRS, à l'UHP, à Nancy 2,
ainsi qu'à l'INPL. Cette équipe travaille essentiellement sur des sujets liés à la médecine et aux neurosciences
computationnelles.
Le but de ses recherches est d'étudier les propriétés et les capacités computationelles des réseaux distribués,
numeriques et adaptatifs, tels qu'observés dans le système neuronal. Dans ce contexte, le but est de
comprendre comment des propriétés complexes émergent de ces systèmes, le tout en restant proche du domaine
d'inspiration que sont les neurosciences. 

\subsection{Membres de l'équipe} % (fold)
\label{sub:Membres de l'équipe}

\subsubsection{Membres permanents} % (fold)
\label{ssub:Membres permanents}

\setlength{\columnseprule}{0.4pt}
\begin{multicols}{2}
\begin{itemize}
   \item \textbf{Responsable d'équipe}
      \begin{itemize}
         \item Frédéric Alexandre
      \end{itemize}
   \item \textbf{Assistants de projet}
      \begin{itemize}
         \item Laurence Benini
         \item Martine Kuhlmann 
      \end{itemize}
   \item \textbf{Chercheurs}
      \begin{itemize}
         \item Axel Hutt
         \item Dominique Martinez
         \item Nicolas Rougier
         \item Thierry Viéville, Sophia Antipolis
         \item Thomas Voegtlin 
      \end{itemize}
   \item \textbf{Personnel universitaire}
      \begin{itemize}
         \item Yann Boniface
         \item Laurent Bougrain
         \item Bernard Girau
      \end{itemize}
   \item \textbf{Collaborateur externe}
      \begin{itemize}
         \item Hervé Frezza-Buet, Supelec
      \end{itemize}
\end{itemize}
\end{multicols}

% subsubsection Membres permanents (end)

\subsubsection{Membres temporaires} % (fold)
\label{ssub:Membres temporaires}

\begin{multicols}{2}
\begin{itemize}
   \item \textbf{Post-doctorants}
      \begin{itemize}
         \item  Jean-Charles Quinton
         \item  Octave Boussaton
      \end{itemize}
   \item \textbf{Étudiants en thèse}
      \begin{itemize}
         \item Lucian Aleçu
         \item Hana Bel Mabrouk
         \item Mauricio Cerda
         \item Georgios Detorakis
         \item Thomas Girod
         \item Mathieu Lefort
         \item Maxime Rio
         \item Carolina Saavedra
         \item Wahiba Taouali
      \end{itemize}
   \item \textbf{Ingénieurs de recherche}
      \begin{itemize}
         \item Mohamed Ghaïth Kaabi
         \item Baptiste Payan
         \item Marie Tonnelier
      \end{itemize}
   \item \textbf{Stagiaire}
      \begin{itemize}
         \item Jenny Hernandez (Laboratorio Nacional de Informática Avanzada(LANIA), Mexique)
         \item Matthieu Maury
      \end{itemize}
\end{itemize}
\end{multicols}

% subsubsection Membres temporaires (end)

% subsection Membres de l'équipe (end)

% section Cortex (end)

% chapter Envirronement (end)
