\chapter{Prédiction des séquences neuronales} % (fold)
\label{cha:Prédire des séquences neuronales}

Comme vu précédemment, il n'y a pas de donné biologique associé à la
\textit{pirouette} du ver, ormis sur son utilité et son rôle dans la recherche
de nourriture chez \celeg{}\cite{Gray2005}. L'une des hypothèse serais que le
circuit de la marche avant et celui de la marche arrière serait actif
simultanément pour réaliser une \textit{pirouette} en effet le cablage du
circuit neural permettrais la modulation de l'activité des motoneurones pour
réaliser une \textit{pirouette}. Malheureusement le cablage seul du circuit ne
permet pas de prédire le fonctionnement, il faudrais des données biologique tel
que des données d'imagerie calcique\footnote{Technique d'observation des
signaux nerveux en reliant le changement de concentration d'ion $Ca^{2+}$ avec
l'activité des neurones.}, ou la possible présence et rôle de neuromodulateur
chez \celeg{}.

Etant donné qu'il est actuellement impossible d'accéder à ce type de donné, une
alternative serais d'utiliser des algorithmes d'apprentissage pour essayer de
prédire des séquences neural inconue, à partir d'un entrainement sur des séquences
connue.

\subsection{Filtre de Kalman} % (fold)
\label{sub:Filtre de Kalman}

Le filter de Kalman est un algorithme d'estimation récursif. Pour mesurer
l'état courant, il se sert de l'état précédent. Ce type de filtre peu être
utiliser pour réaliser du filtrage, du lissage, ou de la prédiction. L'usage
qui nous interesse ici est bien sur la prédiction, mais avec un entrainement
pour les matrices d'observation et de transition utilisées par le script.

En effet, ces matrices doivent être calculées en fonction de l'usage choisis,
dans notre cas, il est plus simple de réaliser un apprentissage pour obtenir de
bonnes matrices

% subsection Filtre de Kalman (end)

% chapter Prédire les séquences neuronales (end)
