\chapter{Prédiction des séquences neuronales} % (fold)
\label{cha:Prédire des séquences neuronales}

Hormis la marche avant est arrière, les autres comportements du ver sont moins
bien connus au niveau neuronal. Très peu de données biologiques sur l'activité
neuronale.

Ainsi, hormis des données sur l'utilité et le rôle de la \textit{pirouette}
dans la recherche de nourriture chez \celeg{}\cite{Gray2005}, aucune
information sur l'activité neurale n'est disponible. L'une des hypothèses
est que le circuit de la marche avant et celui de la marche arrière seraient
actifs simultanément pour réaliser une \textit{pirouette}. En effet le câblage
du circuit neural permettrait la modulation de l'activité des motoneurones pour
réaliser une \textit{pirouette}. Malheureusement le câblage seul du circuit ne
permet pas d'en prédire le fonctionnement, il faudrait des données telles que
des données d'imagerie calcique\footnote{Technique d'observation des signaux
nerveux en reliant le changement de concentration d'ion $Ca^{2+}$ avec
l'activité des neurones.}, ou la possible présence et rôles de
neuromodulateurs\footnote{Les neuromodulateurs sont des molécules qui ne
propagent pas d'information, mais influencent le transfert et/ou la recapture
de certain neurotransmetteurs. Ils peuvent donc radicalement modifier la
manière dont l'information transite.} chez \celeg{}.

Étant donné qu'il est actuellement impossible d'accéder à ce type de donnée, une
alternative serait d'utiliser des algorithmes d'apprentissage pour essayer de
prédire des séquences neurales inconnues, à partir d'un entrainement sur des
séquences connues.

\section{But et étape} % (fold)
\label{sec:But et étape}

Étant donné que la taille des muscles et directement lié à l'activité neuronale
de \celeg{} nous souhaitons utiliser algorithme de type réseau de neurone pour
déduire l'activité neuronale lors de mouvement pour lesquels on ne possède pas
ce type d'information. De plus le modèle du ver utilisé permet à partir d'une
succession de taille de muscle de calculer le mouvement et la configuration
physique du ver. Il n'est donc pas nécessaire de chercher à prédire d'autre
élément du mouvement.

(1) En premier lieu l'algorithme sera testé sur un apprentissage non supervisé
de la variation de la taille des muscles du ver.  Il devra arriver à prédire
avec une marge d'erreur acceptable la taille lors du mouvement du ver. Puis,
une fois ceci fait, (2) nous passerons l'algorithme en mode supervisé pour le
contraindre à générer les séquences neuronales connues lors de mouvement connu.
Le but étant que la qualité de sa prédiction de taille ne change pas, mais que
dans le même temps son état interne soit contraint à correspondre à l'activité
neuronale du ver simulé. (3) Enfin on générera les séquences neuronales lors de
mouvement non connu à l'aide du filtre; et on validera celle-ci en fournissant
au simulateur la séquence neuronale générée et en visualisant si le modèle se
comporte correctement.

L'algorithme choisi pour cet apprentissage est le filtre de Kalman qui est non
supervisé par défaut, mais dont on peut contraindre la représentation interne
pour le rendre supervisé. C'est un algorithme linéaire qui est donc idéal pour
modéliser des systèmes de ressort.

Pour l'étape (1) nous utiliserons le mouvement généré par le simulateur pour
entrainer et tester l'algorithme; de même pour l'étape (2). L'étape 3
demandera l'extraction du mouvement du ver depuis des vidéos de déplacement à
fin de s'assurer que le mouvement du ver est un mouvement réel.

% section But et étape (end)

\section{Filtre de Kalman} % (fold)
\label{sec:Filtre de Kalman}

Le filtre de Kalman est un algorithme d'estimation récursif. Le filtre cherche
à estimer un état interne en fonction des entrées qu'on lui fournit. Cet état
interne peut être de dimension réduite par rapport aux dimensions d'entrée.
Pour mesurer l'état courant, il se sert ainsi de l'état précédent. Ce type de
filtre peut être utilisé pour réaliser du filtrage, du lissage, ou de la
prédiction.  L'usage qui nous intéresse ici est un usage prédictif du filtre.
Étant donné que les données nécessaires à la déduction des séquences neuronales
du ver ne sont pas disponibles pour le mouvement de \textit{pirouette}, nous
souhaitons utiliser un filtre de Kalman pour prédire des séquences neurales
inconnues et comprendre quels neurones pourraient être en activité lors de la
\textit{pirouette} du ver.

Un filtre de Kalman nécessite 2 matrices pour réaliser la prédiction, une
matrice dite générative, et une matrice de transition\cite{Rao1999}. Calculer
ces deux matrices est dépendant du problème que l'on souhaite résoudre. Dans
notre cas il est plus simple de générer ses matrices à partir d'algorithme
d'apprentissage. Rajesh P. N. Rao a développé un algorithme d'apprentissage
pour calculer la matrice générative et la matrice de transition dans le but de
modéliser le cortex visuel\cite{Rao1999}. C'est cet algorithme d'apprentissage
que nous allons utiliser pour réaliser notre filtre de Kalman.

\begin{figure}[ht]
   \begin{center}
      \psfig{width=15cm,figure=pic/kalman_filter.eps}
   \end{center}
   \caption[Schema du filtre de kalman]{Schema du filtre de kalman utilisé par
   Rajesh P. N. Rao. Source Rao, R. P. (1999). \textit{An optimal estimation
   approach to visual perception and learning}. Vision research, 39(11),
   1963-89. \cite{Rao1999}}
   \label{fig:filtre_kalman}
\end{figure}

Dans son article\cite{Rao1999} Rajesh P. N. Rao montre que l'implémentation du
filtre de Kalman permet de prendre en compte une partie des états précédents
pour prédire l'état suivant en cas d'ambiguïté; ainsi qu'il est suffisamment
robuste pour "resynchroniser" ses prédictions en cas de présentation d'élément
inattendu.

Nous verrons dans la suite deux tests qui montrent ces deux capacités du filtre.

\subsection{Équation du filtre} % (fold)
\label{sub:Équation du filtre}

La figure~\ref{fig:filtre_kalman} page~\pageref{fig:filtre_kalman}, représente
le filtre de kalman utilisé par Rajesh P. N. Rao et décrit en détail dans son
article \cite{Rao1999}. Le filtre réalise ses prédictions en calculant

\[ I_{td} = U\bar{r} \]

avec $U$ la matrice générative, $\bar{r}$ vecteur représentant l'état interne
du filtre, et $I_{td}$ la prédiction. Le vecteur d'état interne $\bar{r}$ de
\textit{k-éléments} caractérisant l'entrée donnée au filtre. La taille de ce
vecteur joue sur la qualité de la prédiction réalisée par le filtre.

Le filtre corrige ensuite son estimation de $\bar{r}$ à l'aide de la
différence entre l'entrée suivante $I$, et l'état estimé précédent. Puis le
vecteur $r$ est passé dans son état suivant à l'aide de la matrice $V$ de
transition pour estimer l'état attendu suivant.\\

Dans son algorithme, les calculs de l'\textit{Inverse Covariance} $\Sigma^{-1}$
et de la \textit{Normalization} $N$ sont remplacés par des scalaires bien
choisis %(voir annexes~\ref{cha:Valeur des variables pour l'apprentissage}
%page~\pageref{cha:Valeur des variables pour l'apprentissage} pour les valeurs
%choisies)
pour réduire le temps de calcul nécessaire à la fois à
l'apprentissage et à la prédiction.

% subsection Équation du filtre (end)

\subsection{Apprentissage d'$U$ et $V$} % (fold)
\label{sub:Apprentissage de U et V}

Les matrices $U$ et $V$ sont initialisées aléatoirement en les contraignants
à être au moins orthonormal, l'état interne $r$ lui, est initialisé au vecteur
nul. L'apprentissage est défini par :

\begin{equation}
   \label{eqn:learning_U}
   \hat{U}(t) = \bar{U}(t) + \overbrace{\alpha[I(t) - \bar{U}(t)\hat{r}(t)]\hat{r}(t)^T}^\text{erreur d'apprentissage}
\end{equation}
\begin{equation}
   \label{eqn:learning_V}
   \hat{V}(t-1) = \bar{V}(t-1) + \underbrace{\beta[\hat{r}(t) - r'(t)]\hat{r}(t-1)^T}_\text{erreur d'apprentissage}
\end{equation}

avec $\bar{U}(t) = \hat{U}(t-1)$, $\bar{V}(t-1) = \hat{V}(t-2)$, $\alpha$ et
$\beta$ les coefficients d'apprentissages et $r'(t)$ définis ci-dessous :

\begin{equation}
   \label{eqn:def_r_prime}
   r'(t) =  \bar{V}(t-1)\hat{r}(t-1) + \bar{m}(t-1)
\end{equation}
\begin{equation}
   \label{eqn:def_r_hat}
   \hat{r}(t) = r'(t) + \frac{N_0}{\sigma^2}\bar{U}(t)^T(I(t)-\bar{U}(t)r'(t))
\end{equation}

avec $\frac{N_0}{\sigma^2}$ la normalisation qui sera remplacée par un scalaire lors
de l'apprentissage.

La boucle d'apprentissage est réalisée autant de fois que nécessaire pour que
les erreurs d'apprentissages deviennent non significatives.

% subsection Apprentissage de U et V (end)

% section Filtre de Kalman (end)

\section{Test du filtre} % (fold)
\label{sec:Test du filtre}

\subsection{Séquence simple} % (fold)
\label{sub:Sequence simple}

Cet exemple est tiré de l'article de Rajesh P. N. Rao. Il permet de vérifier si
le filtre prédit correctement une séquence avec une ambiguïté (voir
fig.~\ref{fig:sequence_simple} page~\pageref{fig:sequence_simple}).

\begin{figure}[ht]
   \begin{center}
      \resizebox{150mm}{!}{\includegraphics{pic/graph_sequence_3_5.eps}}
   \end{center}
   \caption[Prédiction de séquence simple]{Séquence simple avec une ambiguïté
   toutes les 2 étapes. Le filtre arrive à déterminer où il en est dans la
   séquence, pour répondre correctement l'étape suivante. Le premier graphe est
   la séquence fournie à prédire, le deuxième est la prédiction du filtre, le
   troisième est l'erreur normalisée entre l'entrée attendue et la prédiction.
   Source personnelle}
   \label{fig:sequence_simple}
\end{figure}

La séquence représente un pixel blanc sur fond noir ayant 3 états possibles.
Position basse, intermédiaire et haute. Lorsqu'il est au milieu, deux choix sont
possibles, soit aller en haut, soit aller en bas. On voit que le filtre arrive
toujours à choisir la bonne position. Ceci est dû à la matrice de transition
$V$ qui va déterminer deux valeurs différentes à l'état interne $r$ pour définir
cette position selon l'état précédent\cite{Rao1999}.

Le filtre a été ici entrainé en présentant une cinquantaine de fois la séquence :
bas, milieu, haut, milieu et l'état interne $r$ est un vecteur de 5 éléments.

% subsection Sequence simple (end)

\subsection{Test simple de resynchronisation} % (fold)
\label{sub:Test simple de resynchronisation}

Ici le filtre a été entrainé de la même manière que pour la
figure~\ref{fig:sequence_simple}, et possède aussi un état interne $r$ de
taille 5; mais la séquence qu'on lui présente ne correspond pas à son
entrainement (fig~\ref{fig:sequence_simple_erreur}
page~\pageref{fig:sequence_simple_erreur}).

On voit ainsi qu'à l'étape 8 le filtre attendait une position basse, alors que
nous lui présentons une position intermédiaire. L'étape 9 est donc entièrement
fausse puisque le filtre prédit la succession basse. Mais le filtre corrige
rapidement ses prédictions en seulement 1 étape supplémentaire.

\begin{figure}[ht]
   \begin{center}
      \resizebox{150mm}{!}{\includegraphics{pic/graph_sequence_3_error_5.eps}}
   \end{center}
   \caption[Prédiction de séquence simple, avec erreur]{Séquence similaire à la
   figure~\ref{fig:sequence_simple}, mais avec une position inattendue.
   L'entrainement est le même que pour la figure~\ref{fig:sequence_simple}, et
   le filtre se remet à prédire assez rapidement correctement. Le premier graphe
   est la séquence fournie à prédire, le deuxième est la prédiction du filtre,
   le troisième est l'erreur normalisée entre l'entrée attendue et la
   prédiction. Source personnelle}
   \label{fig:sequence_simple_erreur}
\end{figure}

% subsection Séquence simple avec surprise (end)

% section Test du filtre (end)

\section{Prédiction sur le ver complet} % (fold)
\label{sec:Prédiction sur le ver complet}

Le filtre de Kalman va être utilisé pour prédire la taille des muscles lors des
mouvements du ver. En premier lieu il va falloir s'assurer que le filtre arrive
à prédire correctement les mouvements du ver en apprentissage non supervisé
(fonctionnement par défaut du filtre de Kalman), et nous tenterons plus tard de
le contraindre à nous donner l'état neuronal lié à l'état des muscles.

\subsection{Mouvement normal avant du ver} % (fold)
\label{sub:Mouvement normal avant du ver}

Une première étape consiste donc à vérifier si le filtre arrive à correctement
prédire l'état des muscles du ver lors d'un mouvement simple. Le mouvement
choisi en l'occurrence est la marche avant obtenue par simulation avec le
modèle du ver.

\begin{figure}[ht]
   \begin{center}
      \resizebox{150mm}{!}{\includegraphics{pic/graph_ver_complet.eps}}
   \end{center}
   \caption[Prédiction de la taille des muscles lors du mouvement simulé
   avant du ver]{Prédiction de la taille des muscles lors du mouvement simulé
   avant du ver. Le premier graphe est la séquence fournie à prédire, le
   deuxième est la prédiction du filtre, le troisième est l'erreur normalisée
   entre l'entrée attendue et la prédiction. Source personnelle}
   \label{fig:sequence_ver_complet}
\end{figure}

Les données d'entrainement pour l'apprentissage correspondent à l'évolution de
la longueur des 98 muscles du ver simulé pour une durée de 30 secondes de
simulation. Ce jeu d'entrainement est présenté entre 3000 et 5000 lors de la
phase d'apprentissage. L'état interne $r$ est ici un vecteur de 12 éléments.
Cette taille de vecteurs a été choisi car elle correspond au nombre d'unités
neuronales du ver.

Le même jeu de donnée que celui d'entrainement est ensuite fourni pour la phase
de prédiction. Sur la figure~\ref{fig:sequence_ver_complet} on peut voir que l'erreur
est grossièrement comprise entre $1\%$ et $7\%$ avec quelques rares endroits où l'erreur
excède $15\%$. Ce que nous considérons comme correct à ce stade de la recherche.

% subsection Mouvement normal avant du ver (end)

\subsection{Mouvement complexe} % (fold)
\label{sub:Mouvement complexe}

\begin{figure}[ht]
   \begin{center}
      \resizebox{150mm}{!}{\includegraphics{pic/graph_ran_3_5.eps}}
   \end{center}
   \caption[Prédiction de la taille des muscles lors d'un mouvement complexe
   simulé]{Prédiction de la taille des muscles lors d'un mouvement complexe
   simulé du ver. Le premier graphe est la séquence fournie à prédire, le
   deuxième est la prédiction du filtre, le troisième est l'erreur normalisée
   entre l'entrée attendue et la prédiction. Le mouvement réalisé ici est une
   succession aléatoire des différents mouvements disponibles dans le simulateur
   (marche avant, relaxation musculaire, torsion dorsale, torsion ventrale. On
   voit que la prédiction suit assez mal les changements rapides, certaine
   "structure" de l'entrée visible sur le graphe sont perdues lors de la
   prédiction. Source personnelle}
   \label{fig:sequence_mouvement_complexe}
\end{figure}

Ensuite nous cherchons à valider que le filtre prédit correctement la taille
des muscles lors de mouvement composé de plusieurs motifs. Le script du ver est
ici modifié pour forcer le réseau de neurones à réaliser une série de mouvement
aléatoirement choisi, et de durée aléatoire. Un premier essaie est présenté
sur la figure \ref{fig:sequence_mouvement_complexe} page
\pageref{fig:sequence_mouvement_complexe}.  Dans cet essai le ver choisi
aléatoirement entre les 4 mouvements disponibles dans le simulateur (marche
avant, relaxation musculaire, torsion ventrale et torsion dorsale), chacun des
mouvements est maintenu durant un temps aléatoirement choisi entre 1,5 et 3
secondes. Ce mouvement est loin de tout mouvement naturel et on voit que le
filtre n'arrive pas à suivre l'enchainement de mouvement trop rapide du ver
simulé. Ceci est visible par le lissage effectué entre l'entrée et la
prédiction.\\


Un deuxième essai a été réalisé cette fois-ci avec un mouvement simulé composé
de seulement deux motifs, marche avant et relaxation musculaire, sur des durées
comprises entre 3 et 6 secondes (voir figure
\ref{fig:sequence_mouvement_complexe_2pattern}, page
\pageref{fig:sequence_mouvement_complexe_2pattern}). Contrairement à l'essai
précédent, le filtre arrive beaucoup mieux à prévoir la taille des muscles, même
si un lissage de l'entrée est présent.

\begin{figure}[ht]
   \begin{center}
      \resizebox{150mm}{!}{\includegraphics{pic/graph_ran_2pattern.eps}}
   \end{center}
   \caption[Prédiction de la taille des muscles lors d'un mouvement avec 2
   motif]{Prédiction de la taille des muscles lors d'un mouvement simulé du ver
   avec 2 motif (marche avant et relaxation musculaire). Le premier graphe est
   la séquence fournie à prédire, le deuxième est la prédiction du filtre, le
   troisième est l'erreur normalisée entre l'entrée attendue et la prédiction.
   Le mouvement est un enchainement de durée aléatoire entre les 2 motifs
   précité. Source personnelle}
   \label{fig:sequence_mouvement_complexe_2pattern}
\end{figure}

% subsection Mouvement complexe (end)

\subsection{Taille de r optimal} % (fold)
\label{sub:Taille de r optimal}

Une question intéressante serait de savoir si il avec un $r$ de 12 éléments il n'y à
pas de perte d'information, et s'il s'agit de la taille minimale de $r$ sans perte
d'information.

Pour cela nous avons réalisé plusieurs fois les expériences précédentes pour obtenir
un maximum de résultat, et nous avons comparé les erreurs e prédiction selon la
taille de $r$.

%\subsubsection{Mouvement simple} % (fold)
%\label{ssub:Mouvement simple}

\begin{figure}[ht]
   \begin{center}
      \resizebox{140mm}{!}{\includegraphics{pic/graph_mean_error.eps}}
   \end{center}
   \caption[Comparaison de l'erreur pour différents $r$ lors du mouvement avant]{
   Comparaison de l'erreur pour différentes tailles de $r$ lors du mouvement avant
   du ver. La moyenne de l'erreur est réalisée pour 16 itérations de la simulation
   pour chaque valeur de $r$. On voit qu'avec un $r > 8$ on obtient une erreur à
   peu de chose près similaire, donc au-delà de 8 éléments pour  il n'y à pas
   d'apport d'information supplémentaire. Source personnelle}
   \label{fig:comparaison_r_mouvement_simple}
\end{figure}

Avec un mouvement simple (en l'occurrence un déplacement avant sur la figure
\ref{fig:comparaison_r_mouvement_simple} page
\pageref{fig:comparaison_r_mouvement_simple}) on voit que pour des tailles de $r$
supérieur à 8 éléments, l'erreur est très peu variable, il n'y a donc plus de différence
en terme d'information pour un $r > 8$. Le mouvement étant très simple est régulier,
la différence d'erreur entre un $r = 4$ et $r > 8$ n'est pas grande. 
Plus de résultats seraient nécessaires pour déterminer quelle taille optimale serait
nécessaire pour $r$. Principalement des études similaires avec des mouvements plus complexes
que la simple marche avant du ver.

% subsubsection Mouvement simple (end)

%\subsubsection{Mouvement complexe} % (fold)
%\label{ssub:Mouvement complexe}

%\begin{figure}[ht]
%   \begin{center}
%      \resizebox{140mm}{!}{\includegraphics{pic/graph_mean_data_complexe.eps}}
%   \end{center}
%   \caption[Comparaison de l'erreur pour différents $r$ lors d'un mouvement 
%   simulé complexe]{Comparaison de l'erreur pour différentes tailles de $r$ lors
%   d'un mouvement simulé complexe avec 4 motifs (marche avant, relaxation musculaire,
%   torsion ventral, torsion dorsal). Malgré le peu de résultat par rapport a l'essai
%   précédent, on voit qu'il n'y à pas vraiment de réduction dans l'erreur moyenne. Le
%   mouvement est certainement trop complexe que le filtre réduise son erreur correctement.
%   Source personnelle}
%   \label{fig:comparaison_r_mouvement_complexe}
%\end{figure}
%
%\begin{figure}[ht]
%   \begin{center}
%      \resizebox{140mm}{!}{\includegraphics{pic/graph_mean_data_complexe_2.eps}}
%   \end{center}
%   \caption[Comparaison de l'erreur pour différents $r$ lors d'un mouvement simulé
%   complexe]{Comparaison de l'erreure pour différentes taille de $r$ lors d'un mouvement
%   simulé avec 2 motifs possible (marche avant et relaxation musculaire)}
%   \label{fig:comparaison_r_mouvement_complexe_2}
%\end{figure}

% subsubsection Mouvement complexe (end)

% subsection Taille de r optimal (end)

% section Ver complet (end)

% chapter Prédire les séquences neuronales (end)
