\chapter{Prédiction des séquences neuronales} % (fold)
\label{cha:Prédire des séquences neuronales}

Comme vu précédemment, il n'y a pas de donnés biologiques associées à la
\textit{pirouette} du ver, hormis sur son utilité et son rôle dans la recherche
de nourriture chez \celeg{}\cite{Gray2005}. L'une des hypothèses serait que le
circuit de la marche avant et celui de la marche arrière seraient actifs
simultanément pour réaliser une \textit{pirouette}. En effet le câblage du
circuit neural permettrait la modulation de l'activité des motoneurones pour
réaliser une \textit{pirouette}. Malheureusement le câblage seul du circuit ne
permet pas d'en prédire le fonctionnement, il faudrait des données telles que
des données d'imagerie calcique\footnote{Technique d'observation des signaux
nerveux en reliant le changement de concentration d'ion $Ca^{2+}$ avec
l'activité des neurones.}, ou la possible présence et rôles de
neuromodulateurs\footnote{Les neuromodulateurs sont des molécules qui ne
propagent pas d'information, mais influencent le transfert et/ou la recapture
de certain neurotransmetteurs. Ils peuvent donc radicalement modifier la
manière dont l'information transite.} chez \celeg{}.

Étant donné qu'il est actuellement impossible d'accéder à ce type de donné, une
alternative serait d'utiliser des algorithmes d'apprentissage pour essayer de
prédire des séquences neurales inconnues, à partir d'un entrainement sur des
séquences connues.

\section{Filtre de Kalman} % (fold)
\label{sec:Filtre de Kalman}

Le filtre de Kalman est un algorithme d'estimation récursif. Pour mesurer
l'état courant, il se sert ainsi de l'état précédent. Ce type de filtre peut
être utilisé pour réaliser du filtrage, du lissage, ou de la prédiction.
L'usage qui nous intéresse ici est un usage prédictif du filtre. Étant donné
que les données nécessaire à la déduction des séquences neuronales du ver ne
sont pas disponible pour le mouvement de \textit{pirouette}, nous souhaitons
utiliser un filtre de Kalman pour prédire des séquences neurales inconnue et
comprendre quels neurones pourraient être en activité lors de la
\textit{pirouette} du ver.

Un filtre de Kalman nécessite 2 matrices pour réaliser la prédiction, une
matrice dite générative, et une matrice de transition\cite{Rao1999}. Calculer
ces deux matrices est dépendant du problème que l'on souhaite résoudre. Dans
notre cas il est plus simple de générer ses matrices à partir d'algorithme
d'apprentissage. Rajesh P. N. Rao à devellopé un algorithme d'apprentissage
pour calculer les matrices générative et de transition dans le but de modéliser
le cortex visuel\cite{Rao1999}. C'est cet algorithme d'apprentissage que nous
allons utiliser pour réaliser notre filtre de Kalman.

\begin{figure}[ht]
   \begin{center}
      \psfig{width=15cm,figure=pic/kalman_filter.eps}
   \end{center}
   \caption[Schema du filtre de kalman]{Schema du filtre de kalman utilisé par
   Rajesh P. N. Rao. Source \cite{Rao1999}}
   \label{fig:filtre_kalman}
\end{figure}

Dans son article\cite{Rao1999} Rajesh P. N. Rao montre que l'implémentation du
filtre de Kalman permet de prendre en compte une partie des états précédents
pour prédire l'état suivant en cas d'ambiguïté; ainsi qu'il est suffisament
robuste pour "resynchroniser" ses prédictions en cas de présentation d'élèment
inattendu.

Nous verrons dans la suite deux tests qui montre ces deux capacité du filtre.

\subsection{Équation du filtre} % (fold)
\label{sub:Équation du filtre}

La figure~\ref{fig:filtre_kalman} page~\pageref{fig:filtre_kalman}, représente
le filtre de kalman utilisé par Rajesh P. N. Rao et décrit en détail dans son
article \cite{Rao1999}. Le filtre réalise ses prédictions en calculant

\[ I_{td} = U\bar{r} \]

avec $U$ la matrice générative, $\bar{r}$ vecteur représentant l'état interne
du filtre, et $I_{td}$ la prédiction. Le vecteur d'état interne $\bar{r}$ de
\textit{k-element} caractérisant l'entrée donnée au filtre. La taille de ce
vecteur joue sur la qualité de la prédiction réalisé par le filtre.

Le filtre corrige ensuite sont estimation de $\bar{r}$ à l'aide de la
difference entre l'entré suivante, $I$ et l'état estimé précédent. Le vecteur
$r$ est ensuite passé dans sont état suivant à l'aide de la matrice $V$ de
transition, est permet de faire passer l'état interne de $\bar{r}(t)$ à
$\bar{r}(t+1)$.\\

Dans son algorithme, les calculs de l'\textit{Inverse Covariance} $\Sigma^{-1}$
et de la \textit{Normalization} $N$ sont remplacés par des scalaires bien
choisis (voir annexes~\ref{cha:Valeur des variables pour l'apprentissage}
page~\pageref{cha:Valeur des variables pour l'apprentissage} pour les valeurs
choisis).  pour réduire le temps de calcul nécessaire à la fois à
l'apprentissage et à la prédictions.

% subsection Équation du filtre (end)

\subsection{Apprentissage de $U$ et $V$} % (fold)
\label{sub:Apprentissage de U et V}

Les matrices $U$ et $V$ sont initialisées aléatoirement en les contraignants
à être au moins orthonormal, et l'état interne $r$ est initialisé au vecteur
nul. L'apprentissage est définis par :

\begin{equation}
   \label{eqn:learning_U}
   \hat{U}(t) = \bar{U}(t) + \overbrace{\alpha[I(t) - \bar{U}(t)\hat{r}(t)]\hat{r}(t)^T}^\text{erreur d'apprentissage}
\end{equation}
\begin{equation}
   \label{eqn:learning_V}
   \hat{V}(t-1) = \bar{V}(t-1) + \underbrace{\beta[\hat{r}(t) - r'(t)]\hat{r}(t-1)^T}_\text{erreur d'apprentissage}
\end{equation}

avec $\bar{U}(t) = \hat{U}(t-1)$, $\bar{V}(t-1) = \hat{V}(t-2)$, $\alpha$ et
$\beta$ les coefficients d'apprentissages et $r'(t)$ définie ci dessous :

\begin{equation}
   \label{eqn:def_r_prime}
   r'(t) =  \bar{V}(t-1)\hat{r}(t-1) + \bar{m}(t-1)
\end{equation}
\begin{equation}
   \label{eqn:def_r_hat}
   \hat{r}(t) = r'(t) + \frac{N_0}{\sigma^2}\bar{U}(t)^T(I(t)-\bar{U}(t)r'(t))
\end{equation}

avec $\frac{N_0}{\sigma^2}$ la normalization qui seras remplacée par un scalaire lors
de l'apprentissage.

La boucle d'apprentissage est réalisée autant de fois que nécessaire pour que
les erreurs d'apprentissages deviennent non significatives.

% subsection Apprentissage de U et V (end)

% section Filtre de Kalman (end)

\section{Test du filtre} % (fold)
\label{sec:Test du filtre}

\subsection{Séquence simple} % (fold)
\label{sub:Sequence simple}

Cet exemple est tiré de l'article de Rajesh P. N. Rao. Il permet de vérifier si
le filtre prédit correctement une séquence avec une ambiguïté (voir
fig.~\ref{fig:sequence_simple} page~\pageref{fig:sequence_simple}).

\begin{figure}[ht]
   \begin{center}
      \resizebox{150mm}{!}{\includegraphics{pic/graph_sequence_3_5.eps}}
   \end{center}
   \caption[Prédiction de séquence simple]{Séquence simple avec une ambiguïté
   toutes les 2 étapes. Le filtre arrive à déterminer où il en est dans dans la
   séquence, pour répondre correctement l'étape suivante. Le premier graph est
   la séquence fournis à prédire, le deuxième est la prédiction du filtre, le
   troisième est l'erreur normalisée entre l'entrée attendu et la prédiction.
   Source personnel}
   \label{fig:sequence_simple}
\end{figure}

La séquence représente un pixel blanc sur fond noir ayant 3 états possibles.
Position basse, intermédiaire et haute. Lorsqu'il est au milieu, deux choix sont
possible, soit aller en haut, soit aller en bas. On vois que le filtre arrive
toujours à choisir la bonne position. Ceci est du à la matrice de transition
$V$ qui va déterminer deux valeurs différentes à l'état interne $r$ pour définir
cette position selon l'état précédent\cite{Rao1999}.

Le filtre à été ici entrainé en présentant une cinquantaine de fois la séquence
bas, milieu, haut, milieu et l'état interne $r$ est un vecteur de dimension 5.

% subsection Sequence simple (end)

\subsection{Test simple de robustesse} % (fold)
\label{sub:Test simple de robustesse}

Ici le filtre à été entrainé de la même manière que pour la
figure~\ref{fig:sequence_simple}, et possède un aussi un état interne $r$ de
dimension 5 mais la séquence qu'on lui présente ne correspond pas à son
entrainement (fig~\ref{fig:sequence_simple_erreur}
page~\pageref{fig:sequence_simple_erreur}).

On voit ainsi qu'à l'étape 8 le filtre attendais une position basse, alors que
nous lui présentont une position intermédiaire. L'étape 9 est donc entièrement
fausse puisque le filtre prédit la succession basse. Mais le filtre corrige
rapidement ses prédiction en seulement 1 étape supplémentaire.

\begin{figure}[ht]
   \begin{center}
      \resizebox{150mm}{!}{\includegraphics{pic/graph_sequence_3_error_5.eps}}
   \end{center}
   \caption[Prédiction de séquence simple, avec erreur]{Séquence similaire à la
   figure~\ref{fig:sequence_simple}, mais avec une position inattendu.
   L'entrainement est le même que pour la figure~\ref{fig:sequence_simple}, et
   le filtre se remet à prédire assez rapidement correctement. Le premier graph
   est la séquence fournis à prédire, le deuxième est la prédiction du filtre,
   le troisième est l'erreur normalisée entre l'entrée attendu et la
   prédiction. Source personnel}
   \label{fig:sequence_simple_erreur}
\end{figure}

% subsection Séquence simple avec surprise (end)

% section Test du filtre (end)

\section{Prédiction sur le ver complet} % (fold)
\label{sec:Prédiction sur le ver complet}

Le filtre de Kalman va être utiliser pour prédire la taille des muscles lors
des mouvements du ver. En effet, l'activité neuronal du ver peut être lié
directement à sont activité musculaire.  En premier lieu il va falloir
s'assurer que le filtre arrive à prédire correctement les mouvements du ver, et
ensuite on tenteras de le contraindre à nous donner l'état neuronal lié à ses
muscles.

\subsection{Mouvement normal avant du ver} % (fold)
\label{sub:Mouvement normal avant du ver}

Une première étape consiste donc à vérifier si le filtre arrive à correctement
prédire l'état des muscles du ver lors d'un mouvement simple. Le mouvement
choisis en l'occurence est la marche avant obtenue par simulation avec le
modèle du ver.

\begin{figure}[ht]
   \begin{center}
      \resizebox{150mm}{!}{\includegraphics{pic/graph_ver_complet.eps}}
   \end{center}
   \caption[Prédiction de de la taille des muscles lors du mouvement simulé
   avant du ver]{Prédiction de la taille des muscles lors du mouvement simulé
   avant du ver. Le premier graph est la séquence fournis à prédire, le
   deuxième est la prédiction du filtre, le troisième est l'erreur normalisée
   entre l'entré attendu et la prédiction. Source pesonnel}
   \label{fig:sequence_ver_complet}
\end{figure}

Les données d'entrainement pour l'apprentissage correspondent à l'évolution de
la longueurs des 98 muscles du ver simulé pour une durée de 30 secondes de
simulation. L'état interne $r$ est ici un vecteur de dimensions 12. Cette dimensions
à été choisis car correspondante au nombre d'unité neuronale du ver.

% subsection Mouvement normal avant du ver (end)

\subsection{Mouvement complexe} % (fold)
\label{sub:Mouvement complexe}

% subsection Mouvement complexe (end)

% section Ver complet (end)

% chapter Prédire les séquences neuronales (end)
