\chapter{Contexte} % (fold)
\label{cha:Contexte}

\section{L'existant} % (fold)
\label{sec:L'existant}

Malgré sa faible complexité le \celeg{} présente des comportements variés.
Ainsi il se déplace en suivant des gradients d'odeurs
\cite{Ferree1999,Gray2005}, il possède differents types et motifs de mouvements
(nage, rampe, avant, arrière, pirouette), il possède un circuit dédié à la
réponse au toucher \cite{Chalfie1985}, et même une forme de mémoire
\cite{Rankin2005a}.

Peu d'études ont été menées sur la dynamique du circuit neural du \celeg{},
principalement car il est actuellement infaisable d'enregistrer la dynamique de
comportement des neurones chez un ver libre de ses mouvements.  Mais il existe
des études qui permettent d'inférer sur le rôle des différents neurones moteurs
lors du déplacement du ver \cite{Yanik2006,Chronis2007,Leifer2011}.  Des études
statistiques sur les différents motifs de déplacement \cite{Gray2005} du ver,
ainsi que des études portant sur l'éléctrophysiologie de certaines cellules
neurales \cite{Mellem2008a,Lockery2009} ont été menées.

% section L'existant (end)

\section{Contexte Biologique} % (fold)
\label{sec:Contexte Biologique}

\caeleg{} (abrégé \celeg{}) est un nématode (ou ver rond) non parasitaire d'un
millimètre de long. De part sa simplicité, et grâce à différentes études, ce
ver est un organisme modèle en génétique; il est ainsi possible de contrôler
finement l'expression de ses gènes en ciblant uniquement une ou plusieurs de
ses cellules.

Parmis ses autres avantages, figure le fait que son système nerveux est
constitué de seulement 302 neurones câblés et qu'il est globalement invariable
entre les différents individus \cite{Boyle2009}.  De plus, une grande partie de
son système nerveux a été cartographié
\cite{Durbin1987,Gray2005,Boyle2009,Varshney2011}.

\subsection{Anatomie du \celeg{}} % (fold)
\label{sub:Anatomie du caeleg}

intro :
   - on ne présenteras pas tous ici, que le necessaire
   - presentation du systéme musculaire

% subsection Anatomie du \caeleg{} (end)

\subsection{Comportement du ver} % (fold)
\label{sub:Comportement du ver}

% subsection Comportement du ver (end)

\subsection{Présentation du circuit neural de locomotion} % (fold)
\label{sub:Présentation du circuit neural de locomotion}

% subsection Présentation du circuit neural de locomotion (end)

% section Contexte Biologique (end)

\section{Modèle du ver} % (fold)
\label{sec:Modèle du ver}

% section Modèle du vers (end)

\section{Contexte informatique} % (fold)
\label{sec:Contexte informatique}

% section Contexte informatique (end)

% chapter Contexte (end)
