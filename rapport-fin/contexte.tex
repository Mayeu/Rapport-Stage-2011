\chapter{Contexte} % (fold)
\label{cha:Contexte}

\section{L'existant} % (fold)
\label{sec:L'existant}

Peu d'études ont été menées sur la dynamique du circuit neural du \celeg{},
principalement car il est actuellement infaisable d'enregistrer la dynamique de
comportement des neurones chez un ver libre de ses mouvements.  Mais il existe
des études qui permettent d'inférer sur le rôle des différents neurones moteurs
lors du déplacement du ver \cite{Yanik2006,Chronis2007,Leifer2011}.  Des études
statistiques sur les différents motifs de déplacement \cite{Gray2005} du ver,
ainsi que des études portant sur l'éléctrophysiologie de certaines cellules
neurales \cite{Mellem2008a,Lockery2009} ont été menées.

% section L'existant (end)

\section{Contexte Biologique} % (fold)
\label{sec:Contexte Biologique}

\caeleg{} (abrégé \celeg{}) est un nématode (ou ver rond) non parasitaire d'un
millimètre de long. De part sa simplicité, ver est un organisme modèle en biologie
et particulièrement en génétique; il est ainsi possible de contrôler
finement l'expression de ses gènes en ne ciblant qu'une seule ou plusieurs de
ses cellules.

\subsection{Anatomie du \celeg{}} % (fold)
\label{sub:Anatomie du caeleg}

%\begin{figure}[ht]
%   \begin{center}
%      \psfig{width=15cm,figure=pic/celegans_anatomy.eps}
%   \end{center}
%   \caption{Anatomy générale du \caeleg{}. A) représentation global du vers,
%   vue selon le flan gauche. B) coupe transversal du vers. Source
%   \cite{Boyle2009}}
%   \label{fig:celegans_anatomy}
%\end{figure}
\begin{center}
   Ici une image de l'anatomie générale du vers
\end{center}
%intro :
%   - on ne présenteras pas tous ici, que le necessaire
%   - presentation du systéme musculaire

Le \celeg{} adulte mesure en générale entre 1 mm et 1,2 mm de long et
possède un corps tubulaire lisse sans anneaux d'un diamètre maximal
avoisinant les 80 µm (fig~\ref{fig:celegans_anatomy},
page~\pageref{fig:celegans_anatomy}).

Un ver adulte et invariablement composé de 959 cellules (en omettant les
cellules qui deviendrons des spermatozoïdes ou des oeufs), incluant 302
neurones et 95 muscles.  \cite{Boyle2009} L'invariabilité et le peu de
cellules composant ce vers en fait donc un organisme idéal pour l'étude
biologique et les simulations informatiques.\\


Comme tous \textit{nématode} n'incluant pas de partie rigide, le \celeg{}
compte sur la pression interne réalisé par ses fluides et organes pour
maintenir sa forme et réaliser ses mouvements.  Comme on peut le voir sur la
figure \ref{fig:celegans_anatomy} B), les muscles sont organisés en quatre
quadrant le long du corps; et contrairement au espèces possédant un
squelette où les muscles ne sont attaché qu'à un seul point des os, les
muscles sont ici entièrement attaché au cuticule. Ceci est important puisque
c'est cette propriété qui permettra la réalisation d'une torsion régulière
et sinusoïdale.

Les deux quadrant de muscles sont innervé par la corde neuronal dorsal, et de
manières similaire, les quadrants ventraux sont innervé par la corde neuronal
ventral. Chaque pair de quadrants (dorsal et ventral) est controlée par le même
jeu de neurones, ainsi le vers ne peut réaliser des torsions que dans un plan à
2 dimensions. A noter que seul la tête du vers (composé des quatres premières
ligne de muscles) est innervé de manière à pouvoir réaliser des mouvements en 3
dimensions. Le ver se déplace ainsi sur ses flancs gauche ou droit, en réalisant
une succession de contraction et décontraction des muscles dorsaux et ventraux
générant un mouvement sinusoïdale.

% subsection Anatomie du \caeleg{} (end)

\subsection{Comportement du ver} % (fold)
\label{sub:Comportement du ver}

Nous nous interesserons ici au comportement de déplacement du vers, mais il est
à noté que malgrès sont faible nombre de neurone le \celeg{} est capable de
suivre ou éviter des gradient de température et/ou d'odeur
\cite{Ferree1999,Gray2005}, d'adapter son comportement si on le touche
(acceleration, fuite) \cite{Chalfie1985} et il possède aussi une forme de
mémoire \cite{Rankin2005a}.\\

% TODO donner la compo du gél en réf
Comme dis précédemment le ver ce déplace en contractant ses muscles dorsaux et
ventraux pour réaliser un mouvement sinusoïdale, tout en reposant sur ses
flancs droit ou gauche. L'étude du mouvement se fait généralement dans des
boites de petri contenant du gel d'agar celui ci leur permettant de reposer
sans s'enfoncer dedans. Le ver rampe ainsi à la surface du gel en suivant une
sinusoïde avec une longueur d'onde aproximative de 2/3 de la longueur du corps
à une fréquence de 0,5 Hz en moyenne\cite{Boyle2009}. Ce mouvement lui permet
sois de se déplacer vers l'avant, sois vers l'arrière. Le \celeg{} à bien sur
la possibilité de moduler ce mouvement pour adapter sa vitesse.

Il est capable de tourner de manière lente pour suivre ou fuire des gradient
d'odeur et de température, mais il présente aussi un mouvement nommé
\textit{pirouette} (ou \textit{Omega-turn}, du à la forme en $\Omega$ qu'il
prend dans ce cas.) qui lui permet de réaliser en changement de direction
brusque. Ce mouvement est souvent utiliser pour passer d'un mouvement en marche
arrière à un mouvement en marche avant, où lorsque le ver recherche sa
nouriture.\\

\begin{center}
   Ici une image du mouvement du vers
\end{center}

% TODO donner la compo du liquide généralement utilisé en réf
Le vers est également capable de nager dans un environnement liquide. La fréquence
de son mouvement passe ainsi à 2 Hz. Différentes études on montrée que le ver
adapte ainsi sont mouvement en fonction des contraintes de son environnement.
Plus d'information peut être trouvé dans la thêse de J. H. Boyle \cite{Boyle2009}

% subsection Comportement du ver (end)

\subsection{Présentation du circuit neural de locomotion} % (fold)
\label{sub:Présentation du circuit neural de locomotion}

Le circuit neural du \celeg{} à été étudié en profondeur, de sorte qu'on
connait non seulement la position de tous ses neurones mais aussi la grande
majorité de ses connections neurales (comme vu précédent celle si sont
invariable d'un ver à l'autre).  Ainsi il est possible d'avoir une carte
précise des liens entre chaque neurone
\cite{Durbin1987,Chen2006,Boyle2009,Varshney2011}.  Plusieurs de ses donnée
peuvent étre trouvée à l'adresse
\url{http://www.wormatlas.org/neuronalwiring.html}, le site \textit{Worm Atlas}
étant une base de donnée structurel et comportementale du \celeg{}.

Si on connait très bien les liens entre chaque neurone, on ne connais en
revanche pas le rôle précis de certains neurones. Je vais présenter ici le
circuit neural utilisé lors de la locomotion avant, ainsi que celui que l'on
considère comme étant générateur du mouvement arrière.

\subsubsection{Présentation d'ensemble} % (fold)
\label{ssub:Présentation d'ensemble}

% TODO ajouter une image du circuit neural générale
\begin{center}
   ici, une image du circuit neural générale, représentant l'anneau neural et la corde ventral
\end{center}

On peut rapidement dicerner deux partie principal dans le circuit neural du
\celeg{}, un ganglion au niveau de la tête du ver, appelé \textit{anneau
nerveux}, et la \textit{corde ventral}.

Les muscles de la tête (quatre premier quadrant de muscles) ne sont innervé que
par \textit{l'anneau nerveux}, le reste des muscles en revanche, est innervé
par \textit{l'anneau nerveux}, et par la \textit{corde ventral}.  Le rôle
présumé de \textit{l'anneau nerveux} est celui de la gestion des comportement
de haut niveaux (poursuite/fuite de gradient, réponse au touché), alors que la
très grande majorité des neurones de la \textit{corde ventral} ne se charge que
de la contraction et décontraction des muscles; et le rôle unitaire des neurones
la composant est mieux connue.

% subsubsection Présentation d'ensemble (end)

\subsubsection{Circuit de la corde ventral} % (fold)
\label{ssub:Circuit de la corde ventral}

Comme dis précédement, chaque paire de muscle présent dans la partie dorsal
(et de manière équivalente dans la partie ventral), et innervé par le même
jeu de neurone, ont les considéreras comme combiné dans la suite de ce document.

Il éxiste 6 jeux de neurones différents dans la \textit{corde ventral} pour la
gestion de la locomotion \cite{Boyle2009}, DA (9), DB (7), DD (6), VA (12), VB
(11), VD (13), ou le chiffre entre parenthése designe le nombre de neurone de
chaque type, et V désignant Ventral, et D dorsal.  Les neurones de type A et B
(DA, DB, VA, VB) sont des neurones excitant, alors que les neurones de type D
(DD, VD) sont des neurones inhibiteur.\\


Il est admis que les neurones de type B sont dédié au déplacement avant du vers,
et que ce même circuit est répété dans l'autre sens par les neurones de type A
pour gérer le déplacement arrière \cite{AltunZ.F.andHall2011,Boyle2009,White1986}.

On peut noter que les zones de jonctions neuromusculaire \footnote{Zone de
contact entre un neurone et un muscle permettant de déclencher un potentiel
d'action pour controler le muscle} ne se chevauche pas, alors que les axones
des neurones se prolonge vers l'arrière pour les neurones de type B (et
inversement vers l'avant pour les types A) et se chevauche entre elle. Cet
allongement joue le rôle de récépteur de torsion pour le vers, et lui permet
ainsi de synchroniser le mouvement de ses muscles \cite{Boyle2009}. A noter que
cette fonction à été proposé par R. L. Russel et L. Byerly, le tout cité par
White et al. \cite{White1986}, mais n'a jamais été vérifié expérimentalement
pour l'instant.


% TODO mettre une image de "l'aternance de DD/VD"
\begin{center}
   Ici, une image de "l'alternance" de VD/DD
\end{center}
A contrario des neurones de type A et B qui n'ont pas de lien entre eux, les
neurones de type D (DD, VD) sont éxcité par les neurones de type A et B, en
plus d'hiniber les muscles. Les neurones DD sont excité par les neurones VA/B,
et les neurones VD par les neurones DA/B. Grâce à cela, lorsque des muscles se
contractes d'un coté, l'autre coté seras relaxé; ceci générant alors une
sinusoïde. La torsion étant elle controlé par les récépteur de torsion qui
prévienne lorsqu'il faut alterner les rôles. Ce phénomène seras appelé
\textit{cross-inhibition} dans la suite.\\


En plus de ces neurones moteurs, ont trouve des interneurones qui ont pour
charge de transmettre l'information entre différent neurones. Je ne
présenterais pas tous ces interneurones, mais seulement 2 paires, AVB/AVA et
PVC/AVD, qui sont pertinent dans le contexte de ce stage.  Ces interneurones
permette la génération de réaction de plus haut niveaux
\cite{Boyle2009,White1986}.  Ainsi la paire AVB/AVA se charge de la marche
avant et la marche arrière, ainsi que le passage de l'un à l'autre. AVB gérant
la marche avant, et AVA la marche arrière. On pense aussi que l'interaction
entre ces deux interneurones permet la génération d'une \textit{pirouette},
mais rien n'à encore étais prouvé que ce soit en terme expérimentale, ou de
simulation.

Les interneurones PVC et AVD quand à eux, sont responsable de la réponse aux
touché du vers. En plus de cette fonction, il peuvent légèrement compenser le
rôle de AVA/AVB s'il sont absent ou dysfonctionnant. Le soma du neurone PVC ce
trouve prés de la queue, alors que AVD se trouve dans la tête du ver.

Les interneurones AVB et PVC sont relié au neurones de type B, et similairement
AVA et AVD sont relié au neurones de type A.

% subsubsection Circuit de la corde ventral (end)

% subsection Présentation du circuit neural de locomotion (end)

% section Contexte Biologique (end)

\section{Modèle du ver} % (fold)
\label{sec:Modèle du ver}

Le modêle choisis pour les simulations est celui détaillé dans la thêse de
Jordan H. Boyle \cite{Boyle2009}. Il à été proposé plusieurs modèle avant
celui ci, mais beaucoup différencié le mouvement dans l'eau du mouvement
dans l'agar. Le modèle devellopé par J. H. Boyle démontre lui qu'il est
possible de simuler les deux comportement avec un seul modèle, uniquement
en ce servant des retours réalisé par l'environement. Il modélise ainsi à la
fois le circuit neural de la corde ventral du ver (ce modèle permet simplement
de faire avancer le vers), le corps du ver (force, élasticité) et les forces
en jeux dans l'environement du ver.

Je vais détaillé ici le modèle final tel qu'il est décrit dans la thêse de J. H. Boyle.

% section Modèle du vers (end)

\section{Contexte informatique} % (fold)
\label{sec:Contexte informatique}

% section Contexte informatique (end)

% chapter Contexte (end)
