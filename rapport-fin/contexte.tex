\chapter{Contexte} % (fold)
\label{cha:Contexte}

%\section{L'existant} % (fold)
%\label{sec:L'existant}

%Peu d'études ont été menées sur la dynamique du circuit neural de \celeg{},
%principalement car il est actuellement impossible d'enregistrer la dynamique de
%comportement des neurones chez un ver libre de ses mouvements.  Mais il existe
%des études qui permettent d'inférer sur le rôle des différents neurones moteurs
%lors du déplacement du ver \cite{Yanik2006,Chronis2007,Leifer2011}.  Des études
%statistiques sur les différents motifs de déplacement \cite{Gray2005} du ver,
%ainsi que des études portant sur l'éléctrophysiologie de certaines cellules
%neurales \cite{Mellem2008a,Lockery2009} ont été menées.

% section L'existant (end)

\section{Contexte Biologique} % (fold)
\label{sec:Contexte Biologique}

\caeleg{} (abrégé \celeg{}) est un nématode (ou ver rond) non parasitaire d'un
millimètre de long. De par sa simplicité, ce ver est un organisme modèle en biologie
et particulièrement en génétique. Il est ainsi possible de contrôler
finement l'expression de ses gènes en ne ciblant qu'une seule ou plusieurs de
ses cellules.

\subsection{Anatomie du \celeg{}} % (fold)
\label{sub:Anatomie du caeleg}

\begin{figure}[ht]
   \begin{center}
      \psfig{width=15cm,figure=pic/celegans_anatomy.eps}
   \end{center}
   \caption[Anatomie générale de \caeleg{}]{Anatomie générale de \caeleg{}. A) coupe longitudinale du ver,
   vue depuis le flanc gauche.B) coupe transversale du ver. Source
   \cite{Boyle2009}}
   \label{fig:celegans_anatomy}
\end{figure}

Un ver \celeg{} adulte mesure en général entre 1 mm et 1,2 mm de long et
possède un corps tubulaire lisse sans anneaux d'un diamètre maximal
avoisinant les 80 µm (fig~\ref{fig:celegans_anatomy},
page~\pageref{fig:celegans_anatomy}).

Un ver adulte est invariablement composé de 959 cellules (en omettant les
cellules qui deviendront des spermatozoïdes ou des oeufs), incluant 302
cellules neuronales et 95 muscles.  \cite{Boyle2009} L'invariabilité et le peu de
cellules composant ce ver en fait un organisme idéal pour l'étude
biologique et les simulations informatiques.\\


Comme tout \textit{nématode} ne possédant pas de partie rigide, \celeg{}
maintient sa forme et peut réaliser des mouvements grâce à ses fluides et
organes qui imposent une pression interne suffisante. Comme on peut le voir sur
la figure \ref{fig:celegans_anatomy} B), les muscles sont organisés en quatre
quadrants le long du corps. Contrairement aux espèces possédant un squelette où
les muscles ne sont attachés qu'à un seul point des os, les muscles sont ici
entièrement attachés à la cuticule. Cette propriété permettra la réalisation
d'une torsion régulière et sinusoïdale.

Les deux quadrants de muscles sont innervés par la corde neuronale dorsale, et
de manière similaire, les quadrants ventraux sont innervés par la corde
neuronale ventrale. Chaque paire de quadrants (dorsale et ventrale) est
contrôlée par le même jeu de neurones, ainsi le ver ne peut réaliser des
torsions que dans un plan à deux dimensions. Contrairement au reste du corp, la
tête du ver, qui est composée des quatre premières lignes de muscles, est
innervée de manière à pouvoir réaliser des mouvements en trois dimensions. Le
ver se déplace sur son flanc gauche ou droit, en réalisant une succession de
contraction et de décontraction des muscles dorsaux et ventraux générant un
mouvement sinusoïdal.

% subsection Anatomie du \caeleg{} (end)

\subsection{Comportement du ver} % (fold)
\label{sub:Comportement du ver}

Nous nous intéresserons ici au comportement de déplacement du ver. Cependant il est
à noter que malgré son faible nombre de neurones, \celeg{} est capable de
suivre ou d'éviter des gradients de température et/ou d'odeur
\cite{Ferree1999,Gray2005}, ainsi que d'adapter son comportement si on le touche
(accélération, fuite) \cite{Chalfie1985}. Il possède également une forme de
mémoire \cite{Rankin2005a}.\\

% TODO donner la compo du gél en réf
Comme dit précédemment, le ver se déplace en contractant ses muscles dorsaux et
ventraux pour réaliser un mouvement sinusoïdal, tout en reposant sur ses
flancs droit ou gauche. L'étude du mouvement se fait généralement dans des
boites de petri contenant du gel d'agar qui lui permet de reposer
sans s'enfoncer. Le ver rampe ainsi à la surface du gel en suivant une
sinusoïde avec une longueur d'onde d'environ 2/3 de la longueur du corps et 
une fréquence de 0,5~Hz en moyenne\cite{Boyle2009}. Ce mouvement lui permet
soit de se déplacer vers l'avant, soit vers l'arrière. \celeg{} a aussi la
possibilité de moduler ce mouvement pour adapter sa vitesse selon ses besoins
(déplacement lent de recherche, déplacement rapide de fuite, etc.).

Il est capable de tourner de manière lente pour suivre ou fuir des gradients
d'odeur et de température, mais il présente également un mouvement nommé
\textit{pirouette} (ou \textit{Omega-turn}, à cause de la forme en $\Omega$ qu'il
prend dans ce cas) et qui lui permet de réaliser un changement de direction
brusque. Ce mouvement est souvent utilisé pour passer d'un mouvement en marche
arrière à un mouvement en marche avant, ou lorsque le ver recherche de la
nourriture.\\

\begin{figure}[ht]
   \begin{center}
      \psfig{width=15cm,figure=pic/celegans_mouvement.eps}
   \end{center}
   \caption[Séquence d'image du mouvement de \celeg{}]{Séquence d'images tirées de deux films illustrant le mouvement de
   \celeg{}. A) Mouvement dans l'eau du ver. B) Mouvement de rampe dans l'agar du ver. Le
   temps indiqué sur chaque image est le temps en seconde. Source \boylecite{}}
   \label{fig:celegans_mouvement}
\end{figure}

% TODO donner la compo du liquide généralement utilisé en réf
Le ver est également capable de nager dans un environnement liquide. La fréquence
de son mouvement passe alors à 2 Hz. Différentes études ont montré que le ver
adapte ainsi son mouvement en fonction des contraintes de son environnement.
Plus d'informations peuvent être trouvées dans la thèse de J. H. Boyle \cite{Boyle2009}

% subsection Comportement du ver (end)

\subsection{Présentation du circuit neural de locomotion} % (fold)
\label{sub:Présentation du circuit neural de locomotion}

Le circuit neural de \celeg{} a été étudié en profondeur. En effet, on connait
non seulement la position de tous ses neurones, mais aussi la grande majorité
de ses connexions neurales (comme vu précédent celles si sont invariables d'un
ver à l'autre).  Ainsi il est possible d'avoir une carte précise des liens
entre chaque neurone \cite{Durbin1987,Chen2006,Boyle2009,Varshney2011}.
Plusieurs de ces données peuvent être trouvées à l'adresse
\url{http://www.wormatlas.org/neuronalwiring.html}, le site \textit{Worm Atlas}
étant une base de données structurelle et comportementale de \celeg{}.

Si on connait très bien les liens entre chaque neurone, on ne connait en
revanche pas le rôle précis de certains neurones. Je vais présenter ici le
circuit neural utilisé lors de la locomotion avant, ainsi que celui
considéré comme étant générateur du mouvement arrière.

\subsubsection{Présentation d'ensemble} % (fold)
\label{ssub:Présentation d'ensemble}

On peut rapidement discerner deux parties principales dans le circuit neural de
\celeg{}, un ganglion au niveau de la tête du ver, appelé \textit{anneau
nerveux}, et la \textit{corde ventrale}.

Les muscles de la tête (quatre premiers quadrants de muscles) ne sont innervés que
par \textit{l'anneau nerveux}, le reste des muscles en revanche est innervé
par \textit{l'anneau nerveux}, et par la \textit{corde ventrale}.  Le rôle
présumé de \textit{l'anneau nerveux} est celui de la gestion des comportements
de haut niveau (poursuite et fuite de gradient, réponse au toucher), alors que la
très grande majorité des neurones de la \textit{corde ventrale} ne se charge que
de la contraction et décontraction des muscles. Le rôle individuel des neurones
composant cette corde est mieux connu que ceux de \textit{l'anneau nerveux}.

% subsubsection Présentation d'ensemble (end)

\subsubsection{Circuit de la corde ventrale} % (fold)
\label{ssub:Circuit de la corde ventrale}

Étant donné que chaque paire de muscles présents dans la partie dorsale (et de
manière équivalente dans la partie ventrale) est innervée par le même jeu de
neurone. C'est pourquoi nous ne les distinguerons pas dans la suite de ce doc

Il existe 6 jeux de neurone différents dans la \textit{corde ventrale} pour la
gestion de la locomotion \cite{Boyle2009}, DA (9), DB (7), DD (6), VA (12), VB
(11), VD (13), où le chiffre entre parenthèses désigne le nombre de neurones de
chaque type, et où V désigne Ventral, et D dorsal.  Les neurones de type A et B
(DA, DB, VA, VB) sont des neurones excitateurs, alors que les neurones de type D
(DD, VD) sont des neurones inhibiteurs.\\


Il est admis que les neurones de type B sont dédiés au déplacement avant du ver,
et que ce circuit est répété dans l'autre sens par les neurones de type A
pour gérer le déplacement arrière \cite{AltunZ.F.andHall2011,Boyle2009,White1986}.

On peut noter que les zones de jonction neuromusculaire \footnote{Zone de
contact entre un neurone et un muscle permettant le transfert d'information 
% Ajouter une footnote
pour contrôler le muscle} ne se chevauchent pas, alors que les processus
des neurones se prolongent vers l'arrière pour les neurones de type B (et
inversement vers l'avant pour les types A) et se chevauchent entre eux. Cet
allongement joue le rôle de récepteur de torsion pour le ver lui permettant
ainsi de synchroniser le mouvement de ses muscles \cite{Boyle2009}. À noter que
cette fonction a été proposée par R. L. Russel et L. Byerly, le tout cité par
White et al. \cite{White1986}, mais n'a jamais été vérifiée expérimentalement
pour l'instant.


\begin{figure}[ht]
   \begin{center}
      \psfig{width=15cm,figure=pic/alternance_dv.eps}
   \end{center}
   \caption[Schéma local du circuit neural de \celeg{}]{Schéma local du circuit neural de \celeg{}. On peut voir
   que lorsque le muscle ventral gauche est excité, le muscle dorsal gauche est
   inhibé; et inversement sur la partie droite du schéma. Les flèches remplies
   représentent des synapses excitatrices (+), alors que les barres représentent des
   synapses inhibitrices (-). En vert : les muscles. En violet : les neurones (avec corps cellulaires et prolongements).
   DD, VA, VB, VC, VD, DA, DB, AS : neurones. Source \textit{Worm Atlas}\cite{AltunZ.F.andHall2011},
   \url{http://www.wormatlas.org/hermaphrodite/nervous/Neuroframeset.html}}
   \label{fig:alternance_dv}
\end{figure}

A contrario des neurones de type A et B qui n'ont pas de lien entre eux, les
neurones de type D (DD, VD) sont tous excités par les neurones de type A et B,
et permettent d'inhiber les muscles. Les neurones DD sont excités par les
neurones VA/B, et les neurones VD par les neurones DA/B. Grâce à cela, lorsque
les muscles d'un côté se contracteront, l'autre côté sera relaxé (voir
fig~\ref{fig:alternance_dv}, page~\pageref{fig:alternance_dv}; ceci générant
alors une sinusoïde. La torsion étant elle contrôlée par les récepteurs de
torsion qui préviennent lorsqu'il faut alterner les rôles. Ce phénomène sera
appelé \textit{cross-inhibition} dans la suite.\\


En plus de ces neurones moteurs, on trouve des interneurones qui ont pour
charge de transmettre l'information entre différents neurones. Je ne
présenterai pas tous ces interneurones, mais seulement 2 paires, AVB/AVA et
PVC/AVD, qui sont pertinent dans le contexte de ce stage. Ces interneurones
permettent la génération de réaction de plus haut niveau
\cite{Boyle2009,White1986}.  Ainsi la paire d'interneurone AVB/AVA se charge de
la marche avant pour AVB, et réciproquement de la marche arrière pour AVA. Ces
interneurones sont certainement utilisé et actif lors du passage de l'un à
l'autre de ces mouvements. On pense aussi que l'interaction entre ces deux
interneurones permet la génération d'une \textit{pirouette}, mais rien n'a
encore été prouvé que ce soit en terme expérimental, ou de simulation.

Les interneurones PVC et AVD quant à eux, sont responsables de la réponse au
touché du ver. En plus de cette fonction, ils peuvent légèrement compenser le
rôle de AVA/AVB s'ils sont absents ou dysfonctionnels. Le corps cellulaire du
neurone PVC se trouve près de la queue, alors que celui d'AVD se trouve dans la
tête du ver.

Les interneurones AVB et PVC sont reliés aux neurones de type B, et AVA et AVD
sont reliés aux neurones de type A.

% subsubsection Circuit de la corde ventrale (end)

% subsection Présentation du circuit neural de locomotion (end)

% section Contexte Biologique (end)

\section{Modèle du ver} % (fold)
\label{sec:Modèle du ver}

Le modèle choisi pour les simulations est celui détaillé dans la thèse de
Jordan H. Boyle \cite{Boyle2009}. Il a été proposé plusieurs modèles avant
celui-ci, mais beaucoup différenciaient le mouvement dans l'eau du mouvement
dans l'agar. J. H. Boyle démontre lui qu'il est possible de simuler les deux
comportements avec un seul modèle, uniquement en se servant des retours
réalisés par l'environnement. Il modélise ainsi à la fois le circuit neural de
la corde ventrale du ver (ce modèle permet simplement de faire avancer le ver),
le corps du ver (force, élasticité) et les forces en jeux dans l'environnement.

Je vais détailler ici le modèle final tel qu'il est décrit dans la thèse de J.
H. Boyle, en ne conservant que la partie dédiée au ver. J'omets donc
volontairement la modélisation de l'environnement. De même que les raisons et
calculs des choix de modélisation se trouvent dans la thèse de Jordan H. Boyle
\cite{Boyle2009}, ils ne seront pas présentés ici.\\

\subsection{Corps de \celeg{}} % (fold)
\label{sub:Corps de celeg}

\begin{figure}[ht]
   \begin{center}
      \psfig{width=15cm,figure=pic/celegans_representation.eps}
   \end{center}
   \caption[Représentation de \celeg{} en simulation]{\celeg{} tel que représenté en simulation. Il est représenté
   par 49 segments rigides (en noir), connectés latéralement par la représentation
   de la cuticule (en rouge), et par les segments diagonaux (en bleu). Source \boylecite{}}
   \label{fig:celegans_representation}
\end{figure}

Le corps du ver est représenté par 49 segments rigides connectés entre eux
latéralement et en diagonale. Les segments latéraux représentent la cuticule du
ver et en reprennent les propriétés élastiques. Les segments diagonaux ont pour
rôle de maintenir les segments rigides, et servent d'approximation pour la
représentation de la pression interne du ver (voir
fig~\ref{fig:celegans_representation}
page~\pageref{fig:celegans_representation}).

Les muscles du ver sont placés au niveau des segments latéraux, chaque segment
étant équivalent à un muscle. Contrairement à la cuticule qui est représentée par un
ressort passif (avec amortissement), les ressorts qui représentent les muscles
changent leur raideur, leur taille au repos et leur amortissement en fonction du niveau
d'activité des muscles. Ce modèle de représentation des muscles permet de gérer
simplement les relations entre longueur et tension ainsi que vitesse et tension
\cite{Boyle2009}.

% subsection Corps du ver (end)

\subsection{Circuit neural} % (fold)
\label{sub:Circuit neural}

Au lieu de considérer exactement le nombre de neurones du ver, on réalise une
approximation avec 12 neurones ventraux et 12 neurones dorsaux de chacun des types
B et D.

\begin{figure}[ht]
   \begin{center}
      \psfig{width=5cm,figure=pic/celegans_segment.eps}
   \end{center}
   \caption[Représentation d'une unité neural de \celeg{}]{Représentation
   schématique d'une des 12 unités neurales qui compose le ver simulé. Chaque
   neurone (cellules rondes et cellules rectangulaires) contrôle 4 muscles
   (cellules hexagonales).  Les synapses excitatrices sont légendées à l'aide
   d'un triangle alors que les synapses inhibitrices le sont à l'aide d'un
   rond.  On observe également les processus neuronaux continuant vers
   l'arrière du ver.  DM: muscle dorsal; DD: neurone dorsal de type D; DB:
   neurone dorsal de type B; VD: neurone ventral de type D; VB: neurone ventral
   de type B; VM : muscle ventral.  Source \boylecite{}}
   \label{fig:celegans_segment}
\end{figure}

On peut ainsi segmenter le ver en 12 unités neurales contenant chacune un neurone DD, VD,
DB, VB, quatre muscles dorsaux et quatre muscles ventraux (voir fig~-\ref{fig:celegans_segment},
page~\pageref{fig:celegans_segment}). Sur la figure \ref{fig:celegans_segment} on peut voir la
cross-inhibition représentée entre les neurones DB en VD, ainsi que VB et DD.

% subsection Circuit neural (end)

% section Modèle du ver (end)

\section{Contexte informatique} % (fold)
\label{sec:Contexte informatique}

\subsection{Implémentation de référence} % (fold)
\label{sub:implémentation de référence}

L'implémentation du modèle réalisé par Jordan H. Boyle servira de référence
pour le comportement simulé du ver. Cette implémentation est réalisée en C++ et
utilise la bibliothèque SUNDIALS \footnote{\textbf{SU}ite of \textbf{N}onlinear and
\textbf{DI}fferential/\textbf{AL}gebraic equation \textbf{S}olvers,
\url{https://computation.llnl.gov/casc/sundials/main.html}} pour la résolution
des équations mécaniques du système. A noter que des différences entre la thèse
et l'implémentation ont été trouvée (voir annexe~\ref{cha:Correction du modèle}
page~\pageref{cha:Correction du modèle}).

Cette version du simulateur implémente tout le système sans utiliser de bibliothèques
annexes (hormis SUNDIALS pour la résolution des équations). L'implémentation initiée
par Thomas Voegtlin se base elle sur un simulateur physique et neural existant.

% subsection implémentation de référence (end)

\subsection{implémentation de Thomas Voegtlin} % (fold)
\label{sub:implémentation de Thomas Voegtlin}

C'est sur cette implémentation que j'ai travaillé. Cette implémentation réalisée
en python, utilise la bibliothèque Brian\cite{Goodman2008} pour calculer les états
des neurones du ver, ainsi que le logiciel SOFA pour simuler la physique et
l'environnement du ver.

L'avantage principal de cette implémentation est de simplifier la simulation
étant donné que Brian et SOFA se charge de réaliser la plupart des calculs, et
qu'il ne reste alors qu'à lier les deux. La communication entre la partie
python et SOFA est réalisée par une bibliothèque créée pour l'occasion, CLONES.

CLONES utilise un socket pour communiquer avec SOFA, et permet de modifier les
propriétés d'objet au sein de SOFA. Dans le cas de la simulation de \celeg{}
cela permet ainsi de modifier les propriétés des muscles, SOFA se chargeant
ensuite de calculer le nouvel état du ver et renvoie les informations au script
python ce qui permet de mettre à jour l'état des récepteurs de torsions du ver.
Le nouvel état des neurones et ensuite calculé, puis celui des muscles, qui
sont repassés à SOFA, etc.

% subsection implémentation de Thomas Voegtlin (end)

\subsection{Différence d'implémentation} % (fold)
\label{sub:Différence d'implémentation}

Le but de Clones est d'offrir une interface générique pour
la simulation entre Brian et SOFA, pour éviter d'avoir à modéliser
le système neurale (Brian) et physique (SOFA) dans toute simulation.

Ceci apporte bien sur des avantages en termes de facilité d'implémentation,
et de temps, mais aussi différents problèmes.

\subsubsection{Changement d'échelle} % (fold)
\label{subs:Changement d'échelle}

SOFA est avant tout dédié à la simulation médicale, et donc à la simulation
à une échelle de taille différente de celle du \celeg{}. Ainsi, dans le code
python représentant le ver, un changement d'échelle était nécessaire pour
faire fonctionner la simulation.

Après recherche au sein du code source de SOFA, il s'avèra que les forces trop
faible sont automatiquement considéré comme insignifiante, de même pour que pour la taille
des ressorts constituant les muscles. Nous avons donc modifier les codes et proposé
ces modifications, et la prise en compte d'autres échelles pour les simulations.

% subsubsection Changement d'échelle (end)

- problème de différence entre la référence est python
- méthode de calcul de SOFA

% subsection Différence d'implémentation (end)

\subsection{Ajout de comportement} % (fold)
\label{sub:Ajout de comportement}

- backward (simple retournement du circuit)
  - gestion du passage forward/backward
- omega turn

% subsection Ajout de comportement (end)

% section Contexte informatique (end)

% chapter Contexte (end)
