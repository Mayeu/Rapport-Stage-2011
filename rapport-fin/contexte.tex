\chapter{Contexte} % (fold)
\label{cha:Contexte}

\section{L'existant} % (fold)
\label{sec:L'existant}

Peu d'études ont été menées sur la dynamique du circuit neural du \celeg{},
principalement car il est actuellement infaisable d'enregistrer la dynamique de
comportement des neurones chez un ver libre de ses mouvements.  Mais il existe
des études qui permettent d'inférer sur le rôle des différents neurones moteurs
lors du déplacement du ver \cite{Yanik2006,Chronis2007,Leifer2011}.  Des études
statistiques sur les différents motifs de déplacement \cite{Gray2005} du ver,
ainsi que des études portant sur l'éléctrophysiologie de certaines cellules
neurales \cite{Mellem2008a,Lockery2009} ont été menées.

% section L'existant (end)

\section{Contexte Biologique} % (fold)
\label{sec:Contexte Biologique}

\caeleg{} (abrégé \celeg{}) est un nématode (ou ver rond) non parasitaire d'un
millimètre de long. De part sa simplicité, ver est un organisme modèle en biologie
et particulièrement en génétique; il est ainsi possible de contrôler
finement l'expression de ses gènes en ne ciblant qu'une seule ou plusieurs de
ses cellules.

\subsection{Anatomie du \celeg{}} % (fold)
\label{sub:Anatomie du caeleg}

\begin{figure}[ht]
   \begin{center}
      \psfig{width=15cm,figure=pic/celegans_anatomy.eps}
   \end{center}
   \caption{Anatomy générale du \caeleg{}. A) représentation global du vers, vue selon le flan gauche. B) coupe transversal du vers. Source \cite{Boyle2009}}
   \label{fig:celegans_anatomy}
\end{figure}

%intro :
%   - on ne présenteras pas tous ici, que le necessaire
%   - presentation du systéme musculaire

Le \celeg{} adulte mesure en générale entre 1 mm et 1,2 mm de long et
possède un corps tubulaire lisse sans anneaux d'un diamètre maximal
avoisinant les 80 µm (fig~\ref{fig:celegans_anatomy},
page~\pageref{fig:celegans_anatomy}).

Un ver adulte et invariablement composé de 959 cellules (en omettant les
cellules qui deviendrons des spermatozoïdes ou des oeufs), incluant 302
neurones et 95 muscles.  \cite{Boyle2009} L'invariabilité et le peu de
cellules composant ce vers en fait donc un organisme idéal pour l'étude
biologique et les simulations informatiques.\\


Comme tous \textit{nématode} n'incluant pas de partie rigide, le \celeg{}
compte sur la pression interne réalisé par ses fluides et organes pour
maintenir sa forme et réaliser ses mouvements.  Comme on peut le voir sur la
figure \ref{fig:celegans_anatomy} B), les muscles sont organisés en quatre
quadrant le long du corps; et contrairement au espèces possédant un
squelette où les muscles ne sont attaché qu'à un seul point des os, les
muscles sont ici entièrement attaché au cuticule. Ceci est important puisque
c'est cette propriété qui permettra la réalisation d'une torsion régulière
et sinusoïdale.

Les deux quadrant de muscles sont innervé par la corde neuronal dorsal, et de
manières similaire, les quadrants ventraux sont innervé par la corde neuronal
ventral. Chaque pair de quadrants (dorsal et ventral) est controlée par le même
jeu de neurones, ainsi le vers ne peut réaliser des torsions que dans un plan à
2 dimensions. A noter que seul la tête du vers (composé des quatres premières
ligne de muscles) est innervé de manière à pouvoir réaliser des mouvements en 3
dimensions. Le ver se déplace ainsi sur ses flancs gauche ou droit, en réalisant
une succession de contraction et décontraction des muscles dorsaux et ventraux
générant un mouvement sinusoïdale.

% subsection Anatomie du \caeleg{} (end)

\subsection{Comportement du ver} % (fold)
\label{sub:Comportement du ver}

Nous nous interesserons ici au comportement de déplacement du vers, mais il est
à noté que malgrès sont faible nombre de neurone le \celeg{} est capable de
suivre ou éviter des gradient de température et/ou d'odeur
\cite{Ferree1999,Gray2005}, d'adapter son comportement si on le touche
(acceleration, fuite) \cite{Chalfie1985} et il possède aussi une forme de
mémoire \cite{Rankin2005a}.\\

Comme dis précédemment le ver ce déplace en contractant ses muscles dorsaux
et ventraux pour réaliser un mouvement sinusoïdale, tout en reposant sur
ses flancs droit ou gauche. L'étude du mouvement se fait généralement 
dans des boites de petri contenant du gel d'agar celui ci leur permettant de
reposer sans s'enfoncer dedans. Le ver rampe ainsi à la surface du gel en suivant
une sinusoïde avec une longueur d'onde aproximative de 2/3 de la longueur du corps
\cite{Boyle2009}

% subsection Comportement du ver (end)

\subsection{Présentation du circuit neural de locomotion} % (fold)
\label{sub:Présentation du circuit neural de locomotion}

% subsection Présentation du circuit neural de locomotion (end)

% section Contexte Biologique (end)

\section{Modèle du ver} % (fold)
\label{sec:Modèle du ver}

% section Modèle du vers (end)

\section{Contexte informatique} % (fold)
\label{sec:Contexte informatique}

% section Contexte informatique (end)

% chapter Contexte (end)
