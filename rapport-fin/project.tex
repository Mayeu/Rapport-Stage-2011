\chapter{Sujet du stage} % (fold)
\label{cha:Sujet du stage}

\section{Énoncé du stage} % (fold)
\label{sec:Énoncé du stage}

Le nématode C. Elegans est un organisme modèle, dont le réseau neuronal a été entièrement décrit sur le plan
anatomique (302 neurones). Cependant, le fonctionnement de ce réseau reste non élucidé. Le nématode présente
divers comportements de base, très stéréotypés : locomotion "forward", "backward", "omega-turn" ou "pirouette"
(voir vidéo). Un modèle informatique et biophysique a permis de démontrer que les retours somatosensoriels
sont fortement impliqués dans l’activité locomotrice ondulatoire (thèse de Jordan H. Boyle). Nous avons
reproduit en partie ces résultats, et nous cherchons à comprendre comment ces comportements sont intégrés.
Le stage consistera à :
\begin{itemize}
   \item se familiariser avec le modèle.
   \item élaborer des hypothèses sur la manière dont sont générés certains comportements (par exemple
      omega-turn), et les tester.
\end{itemize}

% section Énoncé du stage (end)

\section{But du projet} % (fold)
\label{sec:But du projet}

Le projet se base sur le modèle de locomotion avant, décrit dans la thèse de Jordan H. Boyle\cite{Boyle2009}.
Ce modèle représente le ver de façon simplifiée en douze sections comportant le même nombre de muscles et de neurones.
Dans sa thèse, Jordan H. Boyle démontre qu'il est possible, sans modifier le circuit de neurones, de
modéliser les comportements de déplacement avant dans l'eau, dans l'agar et dans un environnement
de viscosité intermédiaire. Contrairement à d'autres modèles qui ne prenaient en compte que le circuit neuronal
du ver, il démontre ainsi que le feedback induit par l'environnement du ver modélisé a un rôle à jouer sur le
fonctionnement du circuit neural de la locomotion.

Le but du projet est d'étendre ce modèle à d'autres comportements de déplacement du ver tel que la locomotion arrière, la pirouette, etc...

% section But du projet (end)

\section{Tâches à réaliser} % (fold)
\label{sec:Tâches à réaliser}

La première partie du travail consiste à se familiariser avec le modèle développé dans la thèse de Jordan
H. Boyle \cite{Boyle2009}, ainsi qu'avec les études menées sur \celeg{}. Il faudra ensuite travailler sur
l'implémentation issue de la thèse. Cette implémentation utilise la bibliothèque Python Brian
(\url{http://www.briansimulator.org/}) ainsi que le logiciel de modélisation physique SOFA
(Simulation Open Framework Architecture, \url{http://www.sofa-framework.org/},\cite{Allard2007}). La liaison entre ces
deux logiciels est réalisée à l'aide de Clones (\textbf{C}losed-\textbf{LO}op \textbf{NE}ural \textbf{S}imulations,\\
\url{http://clones.gforge.inria.fr/}).

Plusieurs autres tâches sont envisageables, parmi celles-ci :
\begin{itemize}
   \item Réalisation et tests de circuits neuronaux pour d'autres mouvements ("backward", "omega-turn")
   \item Implémentation de la chemiotaxie\footnote{Déplacement suivant un gradient de molécule}
   \item autres tâches liées\dots
\end{itemize}

% section Tâches à réaliser (end)

%\section{Planning} % (fold)
%\label{sec:Planning}

% section Planning (end)

% chapter Sujet du stage (end)
