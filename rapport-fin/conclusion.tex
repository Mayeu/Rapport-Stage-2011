\chapter{Conclusion} % (fold)
\label{cha:Conclusion}

La première partie du stage, portant principalement sur l'usage de
SOFA ainsi que sur le fait de rendre le comportement du simulateur exactement similaire à
l'implémentation de référence, aura pris plus de temps que prévu. En effet
nous avons à plusieurs reprise piétiné sur des hypothèses qui au final ne se sont pas
révélées utiles. SOFA tel qu'il est actuellement ne nous permet pas de simuler
le ver de la même manière que l'implémentation de référence, car les solvers inclus dans SOFA ne correspondent pas
à nos besoins. 

Pour les premiers tests de prédiction montrés précédemment, le filtre de Kalman
réalise correctement sa tâche de prédiction avec une marge d'erreur acceptable
au stade actuel des recherches. Le temps a malheureusement manqué pour réaliser
les étapes de travail suivantes, à savoir la contrainte du filtre aux séquences
neuronales du ver pour déterminer les possibles séquences neuronales lors d'autres
déplacements du ver, et la validation des séquences générées par le filtre.
Selon les résultats de ces étapes il faudra ou non, revoir les marges d'erreur
nécessaires au bon fonctionnement du filtre.

% chapter Conclusion (end)
