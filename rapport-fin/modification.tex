\chapter{Modification du modèle} % (fold)
\label{cha:Modification du modèle}

\section{Correction} % (fold)
\label{sec:Correction}

Durant ma phase de recherche documentaire et mes différents essaies avec
SOFA et la simulation, certaine différence sont apparue entre la thèse
et l'implémentation de référence, ainsi j'apporte les corrections suivantes
à la thèse de Jordan H. Boyle \cite{Boyle2009} :
\begin{itemize}
   \item l'équation 8.8 devrait être :
      \begin{equation}
      f^k_{L,m} = \left\{ 
         \begin{array}{rcl}
            \kappa_L(L_{0L,m} - L_{L,m}^k) - \beta_{L}v_{L,m}^k & : & L_{L,m}^k < L_{0L,m}\\
            \kappa_L((L_{0L,m} - L_{L,m}^k) + 16(L_{0L,m} - L_{L,m}^k)^4) - \beta_{L}v_{L,m}^k & : & otherwise
         \end{array}
         \right.
      \end{equation}
   \item l'équation 8.9 devrait être :
      \begin{equation}
      f_{D,m}^k = \kappa_D(L_{0D,m} - L_{D,m}^k) - \beta_{D}v_{D,m}^k
      \end{equation}
   \item l'équation 8.10 devrait être :
      \begin{equation}
      f_{M,m}^k = \kappa_{M,m}^k(L_{0M,m}^k - L_{L,m}^k) - \beta_{M,m}^{k}v_{L,m}^k
      \end{equation}
   \item la définition de $\beta_{0M}$ dans la table 8.1 devrait être $\beta_L \times 100$
\end{itemize}

% section Correction (end)

\section{Différence d'implémentation} % (fold)
\label{sec:Différence d'implémentation}

Le but de Clones est d'offrir une interface générique pour
la simulation entre Brian et SOFA, pour éviter d'avoir à modéliser
le système neurale (Brian) et physique (SOFA) dans toute simulation.

Ceci apporte bien sur des avantages en termes de facilité d'implémentation,
et de temps, mais aussi différents problèmes.

\subsection{Changement d'échelle} % (fold)
\label{sbu:Changement d'échelle}

SOFA est avant tout dédié à la simulation médicale, et donc à la simulation
à une échelle de taille différente de celle du \celeg{}. Ainsi, dans le code
python représentant le ver, un changement d'échelle était nécessaire pour
faire fonctionner la simulation.

Après recherche au sein du code source de SOFA, il s'avèra que les forces trop
faible sont automatiquement considéré comme insignifiante, de même pour que pour la taille
des ressorts constituant les muscles. Nous avons donc modifier les codes et proposé
ces modifications, et la prise en compte d'autres échelles pour les simulations.

% section Changement d'échelle (end)



- problème de différence entre la référence est python
- méthode de calcul de SOFA

% section Différence d'implémentation (end)

\section{Ajout de comportement} % (fold)
\label{sec:Ajout de comportement}

- backward (simple retournement du circuit)
  - gestion du passage forward/backward
- omega turn

% section Ajout de comportement (end)

% chapter Modification du modèle (end)
