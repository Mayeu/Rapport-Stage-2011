\chapter{Sujet du stage} % (fold)
\label{cha:Sujet du stage}

\section{Énoncé du stage} % (fold)
\label{sec:Énoncé du stage}

Le nématode C Elegans est un organisme modèle, dont le réseau neuronal a été entièrement décrit sur le plan
anatomique (302 neurones). Cependant, le fonctionnement de ce réseau reste non élucidé. Le nématode présente
divers comportements de base, très stéréotypés : locomotion "forward", "backward", "omega-turn" ou "pirouette"
(voir vidéo). Un modèle informatique et biophysique a permis de démontrer que les retours somatosensoriels
sont fortement impliqués dans l’activité locomotrice ondulatoire (thèse de Jordan H. Boyle). Nous avons
reproduits en partie ces résultats, et nous cherchons à comprendre comment ces comportements sont intégrés.


Le stage consistera à :
\begin{itemize}
   \item se familiariser avec le modèle.
   \item élaborer des hypothèses sur la manière dont sont générés certains comportements (par exemple
      omega-turn), et les tester.
\end{itemize}

% section Énoncé du stage (end)

\section{But du projet} % (fold)
\label{sec:But du projet}

Le projet se base sur le modèle de locomotion avant décris dans la thèse de Jordan H. Boyle\cite{Boyle2009}.
Ce modèle simplifie le vers en douze sections comportant le même nombre de muscles et de neurones.
Dans sa thése, Jordan H. Boyle démontre qu'il est possible, sans modifier le circuit de neurones, de
modéliser les comportements de déplacement avant dans l'eau, dans l'agar et dans un envirronement
de viscosité intermédiaire. Contrairement à d'autre modèle qui ne prenais en compte que le circuit neural
du vers, il démontre ainsi que le feedback induit par l'environement du vers modelisé à un rôle à jouer sur le
comportement du circuit neural de la locomotion.

Le but du projet est d'étendre ce modèle à d'autre comportement du vers lors de la locomotion.

% section But du projet (end)

\section{Tâches à réaliser} % (fold)
\label{sec:Tâches à réaliser}

La première partie du travail consiste à se familiariser avec le modéle dévellopé dans la thèse de Jordan
H. Boyle \cite{Boyle2009}, ainsi que les études menée sur le \celeg{}. Il faudras ensuite améliorer
l'implémantation issue de la thése. Cette implémentation utilise la bibliothéque Python Brian
(\url{http://www.briansimulator.org/}) ainsi que le logiciel de modélisation physique SOFA
(Simulation Open Framework Architecture, \url{http://www.sofa-framework.org/}). La liaison entre ces
deux logiciels est réalisé à l'aide de Clone (Closed-Loop Neural Simulations,
\url{http://clones.gforge.inria.fr/}).

Plusieurs autre tâches sont envisageable, parmis celle ci :
\begin{itemize}
   \item Réalisation et test de circuit neural pour d'autre mouvement ("backward", "omega-turn")
   \item Implementation de la chimiotaxie
   \item autre tâches en liée\dots
\end{itemize}

% section Tâches à réaliser (end)

%\section{Planning} % (fold)
%\label{sec:Planning}

% section Planning (end)

% chapter Sujet du stage (end)
