\chapter{Sujet du stage} % (fold)
\label{cha:Sujet du stage}

\section{Énoncé du stage} % (fold)
\label{sec:Énoncé du stage}

Le nématode C Elegans est un organisme modèle, dont le réseau neuronal a été entièrement décrit sur le plan
anatomique (302 neurones). Cependant, le fonctionnement de ce réseau reste non élucidé. Le nématode présente
divers comportements de base, très stéréotypés : locomotion "forward", "backward", "omega-turn" ou "pirouette"
(voir vidéo). Un modèle informatique et biophysique a permis de démontrer que les retours somatosensoriels
sont fortement impliqués dans l’activité locomotrice ondulatoire (thèse de Jordan H. Boyle). Nous avons
reproduits en partie ces résultats, et nous cherchons à comprendre comment ces comportements sont intégrés.


Le stage consistera à :
\begin{itemize}
   \item se familiariser avec le modèle.
   \item élaborer des hypothèses sur la manière dont sont générés certains comportements (par exemple
      omega-turn), et les tester.
\end{itemize}

% section Énoncé du stage (end)

\section{Tâches à réaliser} % (fold)
\label{sec:Tâches à réaliser}

% section Tâches à réaliser (end)

\section{Planning} % (fold)
\label{sec:Planning}

% section Planning (end)

% chapter Sujet du stage (end)
