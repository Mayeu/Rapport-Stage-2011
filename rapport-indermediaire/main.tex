%%%%%%%%%%%%%%%%%%%%%%%%%%%%%%%%%%%%%%%%%%%%%%%%%%%%%%%%
%%%%%%                                            %%%%%%
%%%                                                  %%%
%      Modèle de Rapport.                              %
%               Par Matthieu Maury                     %
%                                                      %
%%%                                                  %%%
%%%%%%                                            %%%%%%
%%%%%%%%%%%%%%%%%%%%%%%%%%%%%%%%%%%%%%%%%%%%%%%%%%%%%%%%

\documentclass[11pt,a4paper]{report}
\special{papersize=210mm,297mm} % safe protection against ghostscript multiple version

%%%%%%%%%%%%%%%%%%%%%%%%%%%%%%%%%%%%%%%%%%%%%%%%%%%%%%%%
%% Package essentiel
\usepackage[greek,french]{babel}
\usepackage[T1]{fontenc}
\usepackage{ucs}
\usepackage[utf8x]{inputenc}
\usepackage{fullpage} % pour des marges plus petites

%%%%%%%%%%%%%%%%%%%%%%%%%%%%%%%%%%%%%%%%%%%%%%%%%%%%%%%%
%% Package optionnel
\usepackage{float}
\usepackage{enumerate}
\usepackage{graphicx}
\usepackage{tabularx}
\usepackage{setspace}
\usepackage[dvips]{pstricks}
\usepackage{pstricks-add}
\usepackage{color}
\usepackage{xcolor}
\usepackage{epsfig}
\usepackage{pst-grad} % For gradients
\usepackage{pst-plot} % For axes
\usepackage{amsmath}
\usepackage{amsfonts}
\usepackage{amssymb}
\usepackage{amsxtra}
\usepackage{mathrsfs}
\usepackage{framed}
%\usepackage[framed, thmmarks, amsmath]{ntheorem}
\usepackage{verbatim}
\usepackage{moreverb}
\usepackage{fancyhdr}
\usepackage{url}
\usepackage{listings}
%\usepackage{hyperlinks}
\usepackage{lettrine}

% Try to avoid widow and orphan
\widowpenalty=300
\clubpenalty=300

%%%%%%%%%%%%%%%%%%%%%%%%%%%%%%%%%%%%%%%%%%%%%%%%%%%%%%%%
%%%%%%                                            %%%%%%
%%         Configuration de la mise en page           %%
%%%%%                                              %%%%%
%%%%%%%%%%%%%%%%%%%%%%%%%%%%%%%%%%%%%%%%%%%%%%%%%%%%%%%%

%%%%%%%%%%%%%%%%%%%%%%%%%%%%%%%%%%%%%%%%%%%%%%%%%%%%%%%%
%% Profondeur du sommaire
\setcounter{secnumdepth}{4}
\setcounter{tocdepth}{4}

%%%%%%%%%%%%%%%%%%%%%%%%%%%%%%%%%%%%%%%%%%%%%%%%%%%%%%%%
%% Configuration des chapitres
\makeatletter
\def\@makechapterhead#1{%
\vspace*{50\p@}%
{\parindent \z@ \raggedright \normalfont
\interlinepenalty\@M
\Huge \bfseries\thechapter.\quad#1\par\nobreak
\vskip 20\p@
}}
\makeatother

%%%%%%%%%%%%%%%%%%%%%%%%%%%%%%%%%%%%%%%%%%%%%%%%%%%%%%%%
%%%%%%                                            %%%%%%
%%                Début du Document                   %%
%%%%%                                              %%%%%
%%%%%%%%%%%%%%%%%%%%%%%%%%%%%%%%%%%%%%%%%%%%%%%%%%%%%%%%

\usepackage{caption}
\DeclareCaptionFont{white}{\color{white}}
\DeclareCaptionFormat{listing}{\colorbox{gray}{\parbox{\textwidth}{#1#2#3}}}
\captionsetup[lstlisting]{format=listing,labelfont=white,textfont=white}

\lstset{basicstyle=\fontsize{10}{11}, tabsize=3, frame=b, numbers=left, inputencoding=utf8x, extendedchars=\true, language=C}

\newcommand{\celeg}{\textit{C. Elegans}}
\newcommand{\caeleg}{\textit{Caenorhabditis Elegans}}

\begin{document}

\onehalfspacing % better readability
%\doublespacing % for cheatting purpose :p

%%%%%%%%%%%%%%%%%%%%%%%%%%%%%%%%%%%%%%%%%%%%%%%%%%%%%%%%
%% Inclusion de la page de titre
\pagestyle{fancy}
\renewcommand{\sectionmark}[1]{\markright{\thesection\ #1}}
\renewcommand{\footrulewidth}{0pt}
\renewcommand{\headrulewidth}{0pt}
\fancyhead{} % clear all header fields
\fancyfoot{} % clear all footer fields
\fancyfoot[LO,RE]{\textit{Année Universitaire 2010-2011}}
\fancyfoot[LE,RO]{\textit{Écris avec \LaTeX}}

%\vspace{6cm}

\vspace*{\fill}
\begin{center}
   \textbf{ {\Huge Etude du circuit neural lors de la locomotion chez \caeleg{}}}\\[0.5em]{\huge Rapport de stage de fin d'étude}
\end{center}

\begin{center}
  28 février 2011 - 31 mai 2011
\end{center}

\begin{center}
   Maitre de stage : Thomas Voegtlin, \url{thomas.voegtlin@inria.fr}
\end{center}

\vspace*{\fill}

\newpage


\thispagestyle{empty}

\renewcommand{\footrulewidth}{0.5pt}
\renewcommand{\headrulewidth}{0.5pt}
\fancyhead{} % clear all header fields
\fancyhead[RE,LO]{Rapport intermédiaire de stage}

\fancyhead[RO,LE]{\rightmark}

\fancyfoot{} % clear all footer fields
\fancyfoot[LO,RE]{Matthieu Maury}
\fancyfoot[LE,RO]{\thepage}

%Redéfinition du style fancy - plain, utilisé pour les pages de nouveau chapitre
%Le style par défaut est un style plain
\fancypagestyle{plain}{
\fancyhf{}
\renewcommand{\headrulewidth}{0pt}

%Définition des headers identiques à une page normale
\fancyfoot[LO,RE]{Matthieu Maury}
\fancyfoot[LE,RO]{\thepage}
}

\tableofcontents

%\newpage

\chapter{Envirronement} % (fold)
\label{cha:Envirronement}

\section{L'INRIA} % (fold)
\label{sec:L'INRIA}

% section L'INRIA (end)

\section{LORIA} % (fold)
\label{sec:LORIA}

% section LORIA (end)

\section{Cortex} % (fold)
\label{sec:Cortex}

\subsection{Membres de l'équipe} % (fold)
\label{sub:Membres de l'équipe}

\subsubsection{Membres permanents} % (fold)
\label{ssub:Membres permanents}

\begin{itemize}
   \item \textbf{Responsable d'équipe}
      \begin{itemize}
         \item Frédéric Alexandre
      \end{itemize}
   \item \textbf{Assistant de projet}
      \begin{itemize}
         \item Laurence Benini
         \item Martine Kuhlmann 
      \end{itemize}
   \item \textbf{Chercheurs}
      \begin{itemize}
         \item Axel Hutt
         \item Dominique Martinez
         \item Nicolas Rougier
         \item Thierry Viéville, Sophia Antipolis
         \item Thomas Voegtlin 
      \end{itemize}
   \item \textbf{University faculty}
      \begin{itemize}
         \item Yann Boniface
         \item Laurent Bougrain
         \item Bernard Girau
      \end{itemize}
   \item \textbf{Collaborateurs externe}
      \begin{itemize}
         \item Hervé Frezza-Buet, Supelec
      \end{itemize}
\end{itemize}

% subsubsection Membres permanents (end)

\subsubsection{Membres temporaires} % (fold)
\label{ssub:Membres temporaires}

\begin{itemize}
   \item \textbf{Post-doctorants}
      \begin{itemize}
         \item  Jean-Charles Quinton
         \item  Octave Boussaton
      \end{itemize}
   \item \textbf{Étudiants en thèse}
      \begin{itemize}
         \item Lucian Aleçu
         \item Hana Bel Mabrouk
         \item Mauricio Cerda
         \item Georgios Detorakis
         \item Thomas Girod
         \item Mathieu Lefort
         \item Maxime Rio
         \item Carolina Saavedra
         \item Wahiba Taouali
      \end{itemize}
   \item \textbf{Ingénieurs de recherche}
      \begin{itemize}
         \item Mohamed Ghaïth Kaabi
         \item Baptiste Payan
         \item Marie Tonnelier
      \end{itemize}
   \item \textbf{Stagiaire}
      \begin{itemize}
         \item Jenny Hernandez (Laboratorio Nacional de Informática Avanzada(LANIA), Mexique)
         \item Matthieu Maury
      \end{itemize}
\end{itemize}
% subsubsection Membres temporaires (end)

% subsection Membres de l'équipe (end)

% section Cortex (end)

% chapter Envirronement (end)

\chapter{Sujet du stage} % (fold)
\label{cha:Sujet du stage}

\section{Énoncé du stage} % (fold)
\label{sec:Énoncé du stage}

Le nématode C. Elegans est un organisme modèle, dont le réseau neuronal a été entièrement décrit sur le plan
anatomique (302 neurones). Cependant, le fonctionnement de ce réseau reste non élucidé. Le nématode présente
divers comportements de base, très stéréotypés : locomotion "forward", "backward", "omega-turn" ou "pirouette"
(voir vidéo). Un modèle informatique et biophysique a permis de démontrer que les retours somatosensoriels
sont fortement impliqués dans l’activité locomotrice ondulatoire (thèse de Jordan H. Boyle). Nous avons
reproduits en partie ces résultats, et nous cherchons à comprendre comment ces comportements sont intégrés.

Le stage consistera à :
\begin{itemize}
   \item se familiariser avec le modèle.
   \item élaborer des hypothèses sur la manière dont sont générés certains comportements (par exemple
      omega-turn), et les tester.
\end{itemize}

% section Énoncé du stage (end)

\section{But du projet} % (fold)
\label{sec:But du projet}

Le projet se base sur le modèle de locomotion avant décris dans la thèse de Jordan H. Boyle\cite{Boyle2009}.
Ce modèle simplifie le vers en douze sections comportant le même nombre de muscles et de neurones.
Dans sa thése, Jordan H. Boyle démontre qu'il est possible, sans modifier le circuit de neurones, de
modéliser les comportements de déplacement avant dans l'eau, dans l'agar et dans un environement
de viscosité intermédiaire. Contrairement à d'autres modèles qui ne prenais en compte que le circuit neuronal
du vers, il démontre ainsi que le feedback induit par l'environement du vers modelisé à un rôle à jouer sur le
comportement du circuit neural de la locomotion.

Le but du projet est d'étendre ce modèle à d'autre comportement du vers lors de la locomotion.

% section But du projet (end)

\section{Tâches à réaliser} % (fold)
\label{sec:Tâches à réaliser}

La première partie du travail consiste à se familiariser avec le modèl dévellopé dans la thèse de Jordan
H. Boyle \cite{Boyle2009}, ainsi que les études menée sur le \celeg{}. Il faudras ensuite améliorer
l'implémentation issue de la thése. Cette implémentation utilise la bibliothéque Python Brian
(\url{http://www.briansimulator.org/}) ainsi que le logiciel de modélisation physique SOFA
(Simulation Open Framework Architecture, \url{http://www.sofa-framework.org/}). La liaison entre ces
deux logiciels est réalisée à l'aide de Clones (\textbf{C}losed-\textbf{LO}op \textbf{NE}ural \textbf{S}imulations, 
\url{http://clones.gforge.inria.fr/}).

Plusieurs autres tâches sont envisageables, parmis celle ci :
\begin{itemize}
   \item Réalisation et test de circuit neuronal pour d'autre mouvement ("backward", "omega-turn")
   \item Implémentation de la chimiotaxie\footnote{Déplacement suivant un gradient d'odeur}
   \item autre tâches liée\dots
\end{itemize}

% section Tâches à réaliser (end)

%\section{Planning} % (fold)
%\label{sec:Planning}

% section Planning (end)

% chapter Sujet du stage (end)

\chapter{Avancement} % (fold)
\label{cha:Avancement}

% section L'existant (end)

\section{Implémentation du modèle} % (fold)
\label{sub:Implémentation du modèle}

Dans le but de créer un modèle simple à manipuler et à faire évoluer, une version en Python du
modèle de Jordan H. Boyle a été implémentée avant mon arrivée. Cette version délègue la gestion
des neurones à Brian, la modelisation physique à SOFA et utilise Clones pour les communications
avec SOFA.

\subsection{Modifications et corrections} % (fold)
\label{sub:Modifications et corrections}

En étudiant la thèse de Jordan H. Boyle ainsi que le code source de son modèle, de légéres differences
ont fait leurs apparitions; différences qu'il a fallu répercuter dans notre modèle.
Le modèle actuel se comporte de manière équivalente à celui de Jordan H. Boyle, mais l'implémentation
est légèrement différente. Principalement dû à un problème d'instabilité dans SOFA, nous sommes obligés
de mettre un coefficient multiplicateur pour augmenter l'ordre de grandeur des valeurs physiques du ver
modélisé.

Après étude des algorithmes de calcul utilisés, il apparait que SOFA serait possiblement la cause de
l'instabilité. Le bug a été remonté aux developpeurs
\footnote{\url{https://wiki.sofa-framework.org/tdev/ticket/240}} pour en assurer la prise en compte.

Le but étant, bien sur, d'arriver à se passer de ce coefficient pour être le plus proche possible
du modèle développé par Jordan H. Boyle.

% subsection Modification et correction (end)

% section Implémentation du modéle (end)

\section{Locomotion arrière} % (fold)
\label{sec:Locomotion arrière}

Le circuit neuronal de la locomotion arrière est très proche du circuit neuronal de la locomotion
avant, mais inversé\cite{Boyle2009}. Il n'existe malheureusement que peu d'informations sur le circuit
permettant de passer de la locomotion avant à l'arrière, ormis le cablage du réseau.

Actuellement, l'implémentation du ver possède ce deuxième circuit générant la marche arrière. Mais
le passage de l'un à l'autre n'est pas biologiquement plausible. En s'aidant de données
éléctrophysiologiques sur certains neurones \cite{Mellem2008a,Lockery2009}, ainsi que de différentes
données sur le cablage et les interactions entre neurones
\cite{Chalfie1985,Gray2005,Chen2006,Varshney2011,Leifer2011}, nous developpons un modèle plus
plausible biologiquement pour le passage d'un circuit à l'autre.

% section Locomotion arrière (end)

\section{Pirouette} % (fold)
\label{sec:Pirouette}

La pirouette, ou omega-turn car le mouvement du ver se rapproche d'un $\Omega$, permet au ver
de réaliser un changement de direction soudain. Comparé aux autres motifs de déplacements il y a encore
moins d'information biologique sur le sujet, ormis sur le but et la fréquence de la pirouette dans
la recherche de nourriture \cite{Gray2005}.

Du fait du manque d'informations sur les neurones en jeux et les circuits neuronaux utilisés lors de
la pirouette, il est difficile de la modéliser directement. Par contre nous prévoyons d'utiliser des
vidéos de pirouette et de faire coïncider le mouvement de notre modèle sur le mouvement du ver, et
ainsi remonter de l'état musculaire à l'état neuronal, et voir quels circuits et neurones seraient actifs
lors de la pirouette.

% section Pirouette (end)

% chapter Avancement (end)


\begin{flushleft}
   %\chapter*{Bibliographie}
   \addcontentsline{toc}{chapter}{Bibliographie}
   %\nocite{*}
   \bibliographystyle{plain}
   \bibliography{bibli}
\end{flushleft}

%\newpage
%\setcounter{page}{1}
%\pagenumbering{Roman}
%\appendix
%\include{annexes}

\end{document}
