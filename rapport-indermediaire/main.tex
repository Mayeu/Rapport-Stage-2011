%%%%%%%%%%%%%%%%%%%%%%%%%%%%%%%%%%%%%%%%%%%%%%%%%%%%%%%%
%%%%%%                                            %%%%%%
%%%                                                  %%%
%      Modèle de Rapport.                              %
%               Par Matthieu Maury                     %
%                                                      %
%%%                                                  %%%
%%%%%%                                            %%%%%%
%%%%%%%%%%%%%%%%%%%%%%%%%%%%%%%%%%%%%%%%%%%%%%%%%%%%%%%%

\documentclass[11pt,a4paper]{report}
\special{papersize=210mm,297mm} % safe protection against ghostscript multiple version

%%%%%%%%%%%%%%%%%%%%%%%%%%%%%%%%%%%%%%%%%%%%%%%%%%%%%%%%
%% Package essentiel
\usepackage[greek,french]{babel}
\usepackage[T1]{fontenc}
\usepackage{ucs}
\usepackage[utf8x]{inputenc}
\usepackage{fullpage} % pour des marges plus petites

%%%%%%%%%%%%%%%%%%%%%%%%%%%%%%%%%%%%%%%%%%%%%%%%%%%%%%%%
%% Package optionnel
\usepackage{float}
\usepackage{enumerate}
\usepackage{graphicx}
\usepackage{tabularx}
\usepackage{setspace}
\usepackage[dvips]{pstricks}
\usepackage{pstricks-add}
\usepackage{color}
\usepackage{xcolor}
\usepackage{epsfig}
\usepackage{pst-grad} % For gradients
\usepackage{pst-plot} % For axes
\usepackage{amsmath}
\usepackage{amsfonts}
\usepackage{amssymb}
\usepackage{amsxtra}
\usepackage{mathrsfs}
\usepackage{framed}
%\usepackage[framed, thmmarks, amsmath]{ntheorem}
\usepackage{verbatim}
\usepackage{moreverb}
\usepackage{fancyhdr}
\usepackage{url}
\usepackage{listings}
%\usepackage{hyperlinks}
\usepackage{lettrine}

% Try to avoid widow and orphan
\widowpenalty=300
\clubpenalty=300

%%%%%%%%%%%%%%%%%%%%%%%%%%%%%%%%%%%%%%%%%%%%%%%%%%%%%%%%
%%%%%%                                            %%%%%%
%%         Configuration de la mise en page           %%
%%%%%                                              %%%%%
%%%%%%%%%%%%%%%%%%%%%%%%%%%%%%%%%%%%%%%%%%%%%%%%%%%%%%%%

%%%%%%%%%%%%%%%%%%%%%%%%%%%%%%%%%%%%%%%%%%%%%%%%%%%%%%%%
%% Profondeur du sommaire
\setcounter{secnumdepth}{4}
\setcounter{tocdepth}{4}

%%%%%%%%%%%%%%%%%%%%%%%%%%%%%%%%%%%%%%%%%%%%%%%%%%%%%%%%
%% Configuration des chapitres
\makeatletter
\def\@makechapterhead#1{%
\vspace*{50\p@}%
{\parindent \z@ \raggedright \normalfont
\interlinepenalty\@M
\Huge \bfseries\thechapter.\quad#1\par\nobreak
\vskip 20\p@
}}
\makeatother

%%%%%%%%%%%%%%%%%%%%%%%%%%%%%%%%%%%%%%%%%%%%%%%%%%%%%%%%
%%%%%%                                            %%%%%%
%%                Début du Document                   %%
%%%%%                                              %%%%%
%%%%%%%%%%%%%%%%%%%%%%%%%%%%%%%%%%%%%%%%%%%%%%%%%%%%%%%%

\usepackage{caption}
\DeclareCaptionFont{white}{\color{white}}
\DeclareCaptionFormat{listing}{\colorbox{gray}{\parbox{\textwidth}{#1#2#3}}}
\captionsetup[lstlisting]{format=listing,labelfont=white,textfont=white}

\lstset{basicstyle=\fontsize{10}{11}, tabsize=3, frame=b, numbers=left, inputencoding=utf8x, extendedchars=\true, language=C}

\newcommand{\celeg}{\textit{C. Elegans}}
\newcommand{\caeleg}{\textit{Caenorhabditis Elegans}}

\begin{document}

\onehalfspacing % better readability
%\doublespacing % for cheatting purpose :p

%%%%%%%%%%%%%%%%%%%%%%%%%%%%%%%%%%%%%%%%%%%%%%%%%%%%%%%%
%% Inclusion de la page de titre
\pagestyle{fancy}
\renewcommand{\sectionmark}[1]{\markright{\thesection\ #1}}
\renewcommand{\footrulewidth}{0pt}
\renewcommand{\headrulewidth}{0pt}
\fancyhead{} % clear all header fields
\fancyfoot{} % clear all footer fields
\fancyfoot[LO,RE]{\textit{Année Universitaire 2010-2011}}
\fancyfoot[LE,RO]{\textit{Écris avec \LaTeX}}

%\vspace{6cm}

\vspace*{\fill}
\begin{center}
   \textbf{ {\Huge Etude du circuit neural lors de la locomotion chez \caeleg{}}}\\[0.5em]{\huge Rapport de stage de fin d'étude}
\end{center}

\begin{center}
  28 février 2011 - 31 mai 2011
\end{center}

\begin{center}
   \textbf{Maitre de stage :} Thomas Voegtlin, \url{thomas.voegtlin@inria.fr}
\end{center}

\vspace*{\fill}

\newpage


\thispagestyle{empty}

\renewcommand{\footrulewidth}{0.5pt}
\renewcommand{\headrulewidth}{0.5pt}
\fancyhead{} % clear all header fields
\fancyhead[RE,LO]{Rapport intermédiaire de stage}

\fancyhead[RO,LE]{\rightmark}

\fancyfoot{} % clear all footer fields
\fancyfoot[LO,RE]{Matthieu Maury}
\fancyfoot[LE,RO]{\thepage}

%Redéfinition du style fancy - plain, utilisé pour les pages de nouveau chapitre
%Le style par défaut est un style plain
\fancypagestyle{plain}{
\fancyhf{}
\renewcommand{\headrulewidth}{0pt}

%Définition des headers identiques à une page normale
\fancyfoot[LO,RE]{Matthieu Maury}
\fancyfoot[LE,RO]{\thepage}
}

\tableofcontents

%\newpage

\chapter{Envirronement} % (fold)
\label{cha:Envirronement}

Je réalise mon stage au sein de l'équipe Cortex au \textbf{L}aboratoire L\textbf{o}rrain de
\textbf{R}echerche en \textbf{I}nformatique et ses \textbf{A}pplications (\textbf{LORIA}).
Ce centre de recherche est partagé entre plusieurs institut :
\begin{itemize}
   \item l'\textbf{INRIA}, \textbf{I}nstitut \textbf{N}ational de \textbf{R}echerche en \textbf{I}nformatique
      et en \textbf{A}utomatique (dont je suis dépendant)
   \item le \textbf{CNRS}, \textbf{C}entre \textbf{N}ational de \textbf{R}echerche \textbf{S}cientifique
   \item l'\textbf{INPL}, \textbf{I}nstitut \textbf{N}ational \textbf{P}olytechnique de \textbf{L}orraine
   \item l'\textbf{UHP}, \textbf{U}niversité \textbf{H}enri \textbf{P}oincaré, Nancy 1
   \item \textbf{Nancy 2}, Université Nancy 2
\end{itemize}

\section{L'INRIA} % (fold)
\label{sec:L'INRIA}

L'INRIA est un etablissement de recherche publique à caractère scientifique et technologique. 8 centres de
recherches autonomes sont dispersés dans toute la France (Rocquencourt, Rennes, Sophia Antipolis, Grenoble,
Nancy, Bordeaux, Lille et Saclay).

L'INRIA s'organise autours de 174 équipes-projets à duré de vie limitée (12 ans maximums) et est composée
de 3350 scientifiques\footnote{Source : \url{http://www.inria.fr/institut/inria-en-bref/chiffres-cles}.}.

% section L'INRIA (end)

\section{LORIA} % (fold)
\label{sec:LORIA}

Comme précisé précédemment le LORIA est partagé entre différents instituts de recherches. Tout personnel confondu
c'est 450 personnes qui y travail, incluant 150 chercheurs et 1/3 de doctorants.

Les missions du LORIA sont :
\begin{itemize}
   \item \textbf{Recherche} fondamentale et appliquée au niveau international dans le domaine des Sciences et
      Technologies de l'Information et de la Communication.
   \item \textbf{Formation par la recherche} en partenariat avec les Universités lorraines
   \item \textbf{Transfert technologique} par le biais de partenariats industriels et par l'aide à la création
      d'entreprises
\end{itemize}

% section LORIA (end)

\section{L'équipe Cortex} % (fold)
\label{sec:Cortex}

L'équipe-projet Cortex est affiliée à la fois à l'INRIA, au CNRS, à l'Université Henry Poincaré et Nancy 2,
ainsi que l'INPL. Cette équipe travail essentiellement sur des sujet lié à la médecine et neuroscience
computationnel.

Le but de ces recherche est d'étudier les propriété et les capacités computationel des réseaux distribués,
numeriques et adaptatifs, tel qu'observé dans le systéme neuronal. Dans ce contexte, le but est de
comprendre comment des propriétés complexe émerge de ces systémes, le tout en restant proche du domaine
d'inspiration que son les neurosciences. 

\subsection{Membres de l'équipe} % (fold)
\label{sub:Membres de l'équipe}

\subsubsection{Membres permanents} % (fold)
\label{ssub:Membres permanents}

\setlength{\columnseprule}{0.4pt}
\begin{multicols}{2}
\begin{itemize}
   \item \textbf{Responsable d'équipe}
      \begin{itemize}
         \item Frédéric Alexandre
      \end{itemize}
   \item \textbf{Assistant de projet}
      \begin{itemize}
         \item Laurence Benini
         \item Martine Kuhlmann 
      \end{itemize}
   \item \textbf{Chercheurs}
      \begin{itemize}
         \item Axel Hutt
         \item Dominique Martinez
         \item Nicolas Rougier
         \item Thierry Viéville, Sophia Antipolis
         \item Thomas Voegtlin 
      \end{itemize}
   \item \textbf{University faculty}
      \begin{itemize}
         \item Yann Boniface
         \item Laurent Bougrain
         \item Bernard Girau
      \end{itemize}
   \item \textbf{Collaborateurs externe}
      \begin{itemize}
         \item Hervé Frezza-Buet, Supelec
      \end{itemize}
\end{itemize}
\end{multicols}

% subsubsection Membres permanents (end)

\subsubsection{Membres temporaires} % (fold)
\label{ssub:Membres temporaires}

\begin{multicols}{2}
\begin{itemize}
   \item \textbf{Post-doctorants}
      \begin{itemize}
         \item  Jean-Charles Quinton
         \item  Octave Boussaton
      \end{itemize}
   \item \textbf{Étudiants en thèse}
      \begin{itemize}
         \item Lucian Aleçu
         \item Hana Bel Mabrouk
         \item Mauricio Cerda
         \item Georgios Detorakis
         \item Thomas Girod
         \item Mathieu Lefort
         \item Maxime Rio
         \item Carolina Saavedra
         \item Wahiba Taouali
      \end{itemize}
   \item \textbf{Ingénieurs de recherche}
      \begin{itemize}
         \item Mohamed Ghaïth Kaabi
         \item Baptiste Payan
         \item Marie Tonnelier
      \end{itemize}
   \item \textbf{Stagiaire}
      \begin{itemize}
         \item Jenny Hernandez (Laboratorio Nacional de Informática Avanzada(LANIA), Mexique)
         \item Matthieu Maury
      \end{itemize}
\end{itemize}
\end{multicols}

% subsubsection Membres temporaires (end)

% subsection Membres de l'équipe (end)

% section Cortex (end)

% chapter Envirronement (end)

\chapter{Sujet du stage} % (fold)
\label{cha:Sujet du stage}

\section{Énoncé du stage} % (fold)
\label{sec:Énoncé du stage}

Le nématode C Elegans est un organisme modèle, dont le réseau neuronal a été entièrement décrit sur le plan
anatomique (302 neurones). Cependant, le fonctionnement de ce réseau reste non élucidé. Le nématode présente
divers comportements de base, très stéréotypés : locomotion "forward", "backward", "omega-turn" ou "pirouette"
(voir vidéo). Un modèle informatique et biophysique a permis de démontrer que les retours somatosensoriels
sont fortement impliqués dans l’activité locomotrice ondulatoire (thèse de Jordan H. Boyle). Nous avons
reproduits en partie ces résultats, et nous cherchons à comprendre comment ces comportements sont intégrés.


Le stage consistera à :
\begin{itemize}
   \item se familiariser avec le modèle.
   \item élaborer des hypothèses sur la manière dont sont générés certains comportements (par exemple
      omega-turn), et les tester.
\end{itemize}

% section Énoncé du stage (end)

\section{But du projet} % (fold)
\label{sec:But du projet}

Le projet se base sur le modèle de locomotion avant décris dans la thèse de Jordan H. Boyle\cite{Boyle2009}.
Ce modèle simplifie le vers en douze sections comportant le même nombre de muscles et de neurones.
Dans sa thése, Jordan H. Boyle démontre qu'il est possible, sans modifier le circuit de neurones, de
modéliser les comportements de déplacement avant dans l'eau, dans l'agar et dans un envirronement
de viscosité intermédiaire. Contrairement à d'autre modèle qui ne prenais en compte que le circuit neural
du vers, il démontre ainsi que le feedback induit par l'environement du vers modelisé à un rôle à jouer sur le
comportement du circuit neural de la locomotion.

Le but du projet est d'étendre ce modèle à d'autre comportement du vers lors de la locomotion.

% section But du projet (end)

\section{Tâches à réaliser} % (fold)
\label{sec:Tâches à réaliser}

La première partie du travail consiste à se familiariser avec le modéle dévellopé dans la thèse de Jordan
H. Boyle \cite{Boyle2009}, ainsi que les études menée sur le \celeg{}. Il faudras ensuite améliorer
l'implémantation issue de la thése. Cette implémentation utilise la bibliothéque Python Brian
(\url{http://www.briansimulator.org/}) ainsi que le logiciel de modélisation physique SOFA
(Simulation Open Framework Architecture, \url{http://www.sofa-framework.org/}). La liaison entre ces
deux logiciels est réalisé à l'aide de Clone (Closed-Loop Neural Simulations,
\url{http://clones.gforge.inria.fr/}).

Plusieurs autre tâches sont envisageable, parmis celle ci :
\begin{itemize}
   \item Réalisation et test de circuit neural pour d'autre mouvement ("backward", "omega-turn")
   \item Implementation de la chimiotaxie
   \item autre tâches en liée\dots
\end{itemize}

% section Tâches à réaliser (end)

%\section{Planning} % (fold)
%\label{sec:Planning}

% section Planning (end)

% chapter Sujet du stage (end)

\chapter{Avancement} % (fold)
\label{cha:Avancement}

\caeleg{} (abrégé \celeg{}) est un nématode (ou vers ronds) non parasitaire d'un milimètre de long.
De part sa simplicité, et des différents études menées, ce vers est un organisme modéle en génétique;
il est ainsi possible de controler finement l'expressions de ses gènes en ciblant uniquement une ou
plusieurs de ses cellules.

Parmis ses autres avantages, figure le fait que son système nerveux est constitué de 302
neurones câblés et essentiellement invariable entre les différents individus \cite{Boyle2009}.
De plus, une grande partie de son système nerveux à été cartographié
\cite{Durbin1987,Gray2005,Boyle2009,Varshney2011}.

\section{L'éxistant} % (fold)
\label{sec:L'éxistant}

Malgré sa faible complexité le \celeg{} présente des comportements varié. Ainsi il se déplace
en suivant des gradients d'odeurs \cite{Ferree1999,Gray2005}, posséde differents types et
motifs de mouvement (nage, rampe, avant, arrière, pirouette), posséde un circuit dédié à la 
réponse au toucher \cite{Chalfie1985}, et même une forme de mémoire \cite{Rankin2005a}.

Peu d'étude on étais menée sur la dynamique du circuit neural du \celeg{}, principalement
car il est actuellement infaissable d'enregistrer la dynamique de comportement des neurones
chez un vers libre de mouvement.
Mais il éxiste des études qui permette d'inférer sur le rôle des différents neurones moteurs
lors du déplacement du vers \cite{Yanik2006,Chronis2007,Leifer2011}.
Des études statistiques sur les différents motifs de déplacement \cite{Gray2005} du vers existe, ainsi que
des études portant sur l'éléctrophysiologie de certaines cellules neurales \cite{Mellem2008a,Lockery2009}.

% section L'éxistant (end)

\section{Implémentation du modèle} % (fold)
\label{sub:Implémentation du modèle}

Dans le but de créer un modèle simple à manipuler et à faire évoluer, une version en Python du
modèle de Jordan H. Boyle à été implémentée avant mon arrivée. Cette version délègue la gestion
des neurones à Brian, la modelisation physique à SOFA et utilise Clone pour les communication
avec SOFA.

\subsection{Modification et correction} % (fold)
\label{sub:Modification et correction}

% subsection Modification et correction (end)

% section Implémentation du modéle (end)

\section{Locomotion arrière} % (fold)
\label{sec:Locomotion arrière}

Le circuit neuronal de la locomotion arrière est très proche du circuit neuronal de la locomotion
avant, mais inversé\cite{Boyle2009}. Mais il n'éxistes que peu d'information sur le circuit
permettant de passer de la locomotion avant à arrière, ormis le cablage du réseau.

Actuellement l'implémenttion du vers posséde ce deuxième circuit générant la marche arrière. Mais
le passage de l'un à l'autre n'est pas biologiquement plausible. En s'aidant de donnée
éléctrophysiologique sur certain neurones \cite{Mellem2008a,Lockery2009}, ainsi que différentes
données sur le cablage et les intéractions entre neurones
\cite{Chalfie1985,Gray2005,Chen2006,Varshney2011,Leifer2011}, nous devellopons un modèle plus
plausible biologiquement pour le passage d'un circuit à l'autre.

% section Locomotion arrière (end)

\section{Pirouette} % (fold)
\label{sec:Pirouette}

La pirouette, ou omega-turn car le mouvenment du vers se rapproche d'un $\Omega$, permet au vers
de réaliser un changement de direction soudain. Comparé au autre motif de déplacement il y a encore
moins d'information biologique sur le sujet, ormis sur le but et la fréquence de la pirouette dans
la recherche de nourriture \cite{Gray2005}.

Du fait du manque d'information sur les neurones en jeux et les circuits neuronaux utilisés lors de
la pirouette, il est difficile de la modeliser directement. Par contre nous prévoyons d'utiliser des
vidéos de pirouette et de faire coincider le mouvement du vers avec le mouvement de notre modèle, et
ainsi remonter de l'état musculaire, à l'état neuronal et voir quel circuit et neurones serais actif
lors de la pirouette.

% section Pirouette (end)

% chapter Avancement (end)


\begin{flushleft}
   %\chapter*{Bibliographie}
   \addcontentsline{toc}{chapter}{Bibliographie}
   %\nocite{*}
   \bibliographystyle{plain}
   \bibliography{bibli}
\end{flushleft}

%\newpage
%\setcounter{page}{1}
%\pagenumbering{Roman}
%\appendix
%\include{annexes}

\end{document}
