\chapter{Envirronement} % (fold)
\label{cha:Envirronement}

Je réalise mon stage au sein de l'équipe Cortex au \textbf{L}aboratoire L\textbf{o}rrain de
\textbf{R}echerche en \textbf{I}nformatique et ses \textbf{A}pplications (\textbf{LORIA}).
Ce centre de recherche est partagé entre plusieurs institut :
\begin{itemize}
   \item l'\textbf{INRIA}, \textbf{I}nstitut \textbf{N}ational de \textbf{R}echerche en \textbf{I}nformatique
      et en \textbf{A}utomatique (dont je suis dépendant)
   \item le \textbf{CNRS}, \textbf{C}entre \textbf{N}ational de \textbf{R}echerche \textbf{S}cientifique
   \item l'\textbf{INPL}, \textbf{I}nstitut \textbf{N}ational \textbf{P}olytechnique de \textbf{L}orraine
   \item l'\textbf{UHP}, \textbf{U}niversité \textbf{H}enri \textbf{P}oincaré, Nancy 1
   \item \textbf{Nancy 2}, Université Nancy 2
\end{itemize}

\section{L'INRIA} % (fold)
\label{sec:L'INRIA}

L'INRIA est un etablissement de recherche publique à caractère scientifique et technologique. 8 centres de
recherches autonomes sont dispersés dans toute la France (Rocquencourt, Rennes, Sophia Antipolis, Grenoble,
Nancy, Bordeaux, Lille et Saclay).

L'INRIA s'organise autours de 174 équipes-projets à duré de vie limitée (12 ans maximums) et est composée
de 3350 scientifiques\footnote{Source : \url{http://www.inria.fr/institut/inria-en-bref/chiffres-cles}.}.

% section L'INRIA (end)

\section{LORIA} % (fold)
\label{sec:LORIA}

Comme précisé précédemment le LORIA est partagé entre différents instituts de recherches. Tout personnel confondu
c'est 450 personnes qui y travail, incluant 150 chercheurs et 1/3 de doctorants.

Les missions du LORIA sont :
\begin{itemize}
   \item \textbf{Recherche} fondamentale et appliquée au niveau international dans le domaine des Sciences et
      Technologies de l'Information et de la Communication.
   \item \textbf{Formation par la recherche} en partenariat avec les Universités lorraines
   \item \textbf{Transfert technologique} par le biais de partenariats industriels et par l'aide à la création
      d'entreprises
\end{itemize}

% section LORIA (end)

\section{L'équipe Cortex} % (fold)
\label{sec:Cortex}

L'équipe-projet Cortex est affiliée à la fois à l'INRIA, au CNRS, à l'Université Henry Poincaré et Nancy 2,
ainsi que l'INPL. Cette équipe travail essentiellement sur des sujet lié à la médecine et neuroscience
computationnel.

Le but de ces recherche est d'étudier les propriété et les capacités computationel des réseaux distribués,
numeriques et adaptatifs, tel qu'observé dans le systéme neuronal. Dans ce contexte, le but est de
comprendre comment des propriétés complexe émerge de ces systémes, le tout en restant proche du domaine
d'inspiration que son les neurosciences. 

\subsection{Membres de l'équipe} % (fold)
\label{sub:Membres de l'équipe}

\subsubsection{Membres permanents} % (fold)
\label{ssub:Membres permanents}

\setlength{\columnseprule}{0.4pt}
\begin{multicols}{2}
\begin{itemize}
   \item \textbf{Responsable d'équipe}
      \begin{itemize}
         \item Frédéric Alexandre
      \end{itemize}
   \item \textbf{Assistant de projet}
      \begin{itemize}
         \item Laurence Benini
         \item Martine Kuhlmann 
      \end{itemize}
   \item \textbf{Chercheurs}
      \begin{itemize}
         \item Axel Hutt
         \item Dominique Martinez
         \item Nicolas Rougier
         \item Thierry Viéville, Sophia Antipolis
         \item Thomas Voegtlin 
      \end{itemize}
   \item \textbf{University faculty}
      \begin{itemize}
         \item Yann Boniface
         \item Laurent Bougrain
         \item Bernard Girau
      \end{itemize}
   \item \textbf{Collaborateurs externe}
      \begin{itemize}
         \item Hervé Frezza-Buet, Supelec
      \end{itemize}
\end{itemize}
\end{multicols}

% subsubsection Membres permanents (end)

\subsubsection{Membres temporaires} % (fold)
\label{ssub:Membres temporaires}

\begin{multicols}{2}
\begin{itemize}
   \item \textbf{Post-doctorants}
      \begin{itemize}
         \item  Jean-Charles Quinton
         \item  Octave Boussaton
      \end{itemize}
   \item \textbf{Étudiants en thèse}
      \begin{itemize}
         \item Lucian Aleçu
         \item Hana Bel Mabrouk
         \item Mauricio Cerda
         \item Georgios Detorakis
         \item Thomas Girod
         \item Mathieu Lefort
         \item Maxime Rio
         \item Carolina Saavedra
         \item Wahiba Taouali
      \end{itemize}
   \item \textbf{Ingénieurs de recherche}
      \begin{itemize}
         \item Mohamed Ghaïth Kaabi
         \item Baptiste Payan
         \item Marie Tonnelier
      \end{itemize}
   \item \textbf{Stagiaire}
      \begin{itemize}
         \item Jenny Hernandez (Laboratorio Nacional de Informática Avanzada(LANIA), Mexique)
         \item Matthieu Maury
      \end{itemize}
\end{itemize}
\end{multicols}

% subsubsection Membres temporaires (end)

% subsection Membres de l'équipe (end)

% section Cortex (end)

% chapter Envirronement (end)
