\chapter{Avancement} % (fold)
\label{cha:Avancement}

\caeleg{} (abrégé \celeg{}) est un nématode (ou vers ronds) non parasitaire d'un milimètre de long.
De part sa simplicité, et des différentes études menées, ce vers est un organisme modéle en génétique;
il est ainsi possible de controler finement l'expressions de ses gènes en ciblant uniquement une ou
plusieurs de ses cellules.

Parmis ses autres avantages, figure le fait que son système nerveux est constitué de 302
neurones câblés et essentiellement invariable entre les différents individus \cite{Boyle2009}.
De plus, une grande partie de son système nerveux à été cartographié
\cite{Durbin1987,Gray2005,Boyle2009,Varshney2011}.

\section{L'éxistant} % (fold)
\label{sec:L'éxistant}

Malgré sa faible complexité le \celeg{} présente des comportements variés. Ainsi il se déplace
en suivant des gradients d'odeurs \cite{Ferree1999,Gray2005}, posséde differents types et
motifs de mouvement (nage, rampe, avant, arrière, pirouette), posséde un circuit dédié à la 
réponse au toucher \cite{Chalfie1985}, et même une forme de mémoire \cite{Rankin2005a}.

Peu d'étude on étais menée sur la dynamique du circuit neural du \celeg{}, principalement
car il est actuellement infaissable d'enregistrer la dynamique de comportement des neurones
chez un vers libre de mouvement.
Mais il éxiste des études qui permettent d'inférer sur le rôle des différents neurones moteurs
lors du déplacement du vers \cite{Yanik2006,Chronis2007,Leifer2011}.
Des études statistiques sur les différents motifs de déplacement \cite{Gray2005} du vers, ainsi que
des études portant sur l'éléctrophysiologie de certaines cellules neurales \cite{Mellem2008a,Lockery2009}.

% section L'éxistant (end)

\section{Implémentation du modèle} % (fold)
\label{sub:Implémentation du modèle}

Dans le but de créer un modèle simple à manipuler et à faire évoluer, une version en Python du
modèle de Jordan H. Boyle à été implémentée avant mon arrivée. Cette version délègue la gestion
des neurones à Brian, la modelisation physique à SOFA et utilise Clones pour les communication
avec SOFA.

\subsection{Modification et correction} % (fold)
\label{sub:Modification et correction}

En étudiant la thése de Jordan H. Boyle ainsi que le code source de son modèle, de légéres differences
on fait leurs apparitions; différences qu'il à fallu répercuter dans notre modèle.
Le modèle actuel se comporte de manière équivalente à celui de Jordan H. Boyle, mais l'implémentation
est légèrement différente. Principalement du à un problème d'instabilité dans SOFA, nous sommes obligé
de mettre un coefficient multiplicateur pour augmenter l'ordre de grandeur des valeurs physiques du vers
modélisé.

Après étude des algorithmes de calcul utilisés, il apparait que SOFA serais possiblement la cause de
l'instabilité. Le bug à été remonté aux devellopeurs
\footnote{\url{https://wiki.sofa-framework.org/tdev/ticket/240}} pour en assurer la prise en compte.

Le but étant, bien sur, d'arriver à se passer de ce coefficient pour ètre le plus proche possible
du modèle dévellopé par Jordan H. Boyle.

% subsection Modification et correction (end)

% section Implémentation du modéle (end)

\section{Locomotion arrière} % (fold)
\label{sec:Locomotion arrière}

Le circuit neuronal de la locomotion arrière est très proche du circuit neuronal de la locomotion
avant, mais inversé\cite{Boyle2009}. Il n'éxistes malheureusement que peu d'information sur le circuit
permettant de passer de la locomotion avant à arrière, ormis le cablage du réseau.

Actuellement l'implémenttion du vers posséde ce deuxième circuit générant la marche arrière. Mais
le passage de l'un à l'autre n'est pas biologiquement plausible. En s'aidant de donnée
éléctrophysiologique sur certain neurones \cite{Mellem2008a,Lockery2009}, ainsi que différentes
données sur le cablage et les intéractions entre neurones
\cite{Chalfie1985,Gray2005,Chen2006,Varshney2011,Leifer2011}, nous devellopons un modèle plus
plausible biologiquement pour le passage d'un circuit à l'autre.

% section Locomotion arrière (end)

\section{Pirouette} % (fold)
\label{sec:Pirouette}

La pirouette, ou omega-turn car le mouvenment du vers se rapproche d'un $\Omega$, permet au vers
de réaliser un changement de direction soudain. Comparé au autre motif de déplacement il y a encore
moins d'information biologique sur le sujet, ormis sur le but et la fréquence de la pirouette dans
la recherche de nourriture \cite{Gray2005}.

Du fait du manque d'information sur les neurones en jeux et les circuits neuronaux utilisés lors de
la pirouette, il est difficile de la modeliser directement. Par contre nous prévoyons d'utiliser des
vidéos de pirouette et de faire coincider le mouvement de notre modèle sur le mouvement du vers, et
ainsi remonter de l'état musculaire, à l'état neuronal et voir quel circuit et neurones serais actif
lors de la pirouette.

% section Pirouette (end)

% chapter Avancement (end)
