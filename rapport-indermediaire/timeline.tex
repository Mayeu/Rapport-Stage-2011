\chapter{Avancement}

\section{Avancement actuel}
\section{Prévision}

\caeleg{} (abrégé \celeg{}) est un nématode (ou vers ronds) non parasitaire d'un milimètre de long.
De part sa simplicité, et des différents études menées, ce vers est un organisme modéle en génétique;
il est ainsi possible de controler finement l'expressions de ses gènes en ciblant uniquement une ou
plusieurs de ses cellules.

Parmis ses autres avantages, figure le fait que son système nerveux est constitué de 302
neurones câblés et essentiellement invariable entre les différents individus \cite{Boyle2009}.
De plus, une grande partie de son système nerveux à été cartographié
\cite{Durbin1987,Gray2005,Boyle2009,Varshney2011}.


\section{L'éxistant}
Malgré sa faible complexité le \celeg{} présente des comportements varié. Ainsi il se déplace
en suivant des gradients d'odeurs \cite{Ferree1999,Gray2005}, posséde differents types et
motifs de mouvement (nage, rampe, avant, arrière, pirouette), posséde un circuit dédié à la 
réponse au toucher \cite{Chalfie1985}, et même une forme de mémoire \cite{Rankin2005a}.


Peu d'étude on étais menée sur la dynamique du circuit neural du \celeg{}, et elle porte
toute principalement sur l'étude du déplacement avant du vers \cite{Bryden2008,Boyle2009};
cela est du à l'infaisabilité actuel d'enregistrer la dynamique des neurones chez un vers
se déplaçant librement. Mais il éxiste des études qui permette d'inférer sur le rôle et la dynamique
des différents neurones moteurs lors du déplacement du vers \cite{Yanik2006,Chronis2007,Leifer2011},
des études statistiques sur ses différents motifs de déplacement \cite{Gray2005}, ainsi que
des études portant sur l'éléctrophysiologie de certaines cellules neurales \cite{Mellem2008a,Lockery2009}.

\section{But du projet}

