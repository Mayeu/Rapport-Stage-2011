\chapter{Avancement} % (fold)
\label{cha:Avancement}

\caeleg{} (abrégé \celeg{}) est un nématode (ou ver rond) non parasitaire d'un millimètre de long.
De part sa simplicité, et grâce à différentes études, ce ver est un organisme modèle en génétique;
il est ainsi possible de contrôler finement l'expression de ses gènes en ciblant uniquement une ou
plusieurs de ses cellules.

Parmis ses autres avantages, figure le fait que son système nerveux est constitué de seulement 302
neurones câblés et qu'il est globalement invariable entre les différents individus \cite{Boyle2009}.
De plus, une grande partie de son système nerveux a été cartographié
\cite{Durbin1987,Gray2005,Boyle2009,Varshney2011}.

\section{L'existant} % (fold)
\label{sec:L'existant}

Malgré sa faible complexité le \celeg{} présente des comportements variés. Ainsi il se déplace
en suivant des gradients d'odeurs \cite{Ferree1999,Gray2005}, il possède differents types et
motifs de mouvements (nage, rampe, avant, arrière, pirouette), il possède un circuit dédié à la 
réponse au toucher \cite{Chalfie1985}, et même une forme de mémoire \cite{Rankin2005a}.

Peu d'études ont été menées sur la dynamique du circuit neural du \celeg{}, principalement
car il est actuellement infaisable d'enregistrer la dynamique de comportement des neurones
chez un ver libre de ses mouvements.
Mais il existe des études qui permettent d'inférer sur le rôle des différents neurones moteurs
lors du déplacement du ver \cite{Yanik2006,Chronis2007,Leifer2011}.
Des études statistiques sur les différents motifs de déplacement \cite{Gray2005} du ver, ainsi que
des études portant sur l'éléctrophysiologie de certaines cellules neurales \cite{Mellem2008a,Lockery2009} ont été menées.

% section L'existant (end)

\section{Implémentation du modèle} % (fold)
\label{sub:Implémentation du modèle}

Dans le but de créer un modèle simple à manipuler et à faire évoluer, une version en Python du
modèle de Jordan H. Boyle a été implémentée avant mon arrivée. Cette version délègue la gestion
des neurones à Brian, la modelisation physique à SOFA et utilise Clones pour les communications
avec SOFA.

\subsection{Modifications et corrections} % (fold)
\label{sub:Modifications et corrections}

En étudiant la thèse de Jordan H. Boyle ainsi que le code source de son modèle, de légéres differences
ont fait leurs apparitions; différences qu'il a fallu répercuter dans notre modèle.
Le modèle actuel se comporte de manière équivalente à celui de Jordan H. Boyle, mais l'implémentation
est légèrement différente. Principalement dû à un problème d'instabilité dans SOFA, nous sommes obligés
de mettre un coefficient multiplicateur pour augmenter l'ordre de grandeur des valeurs physiques du ver
modélisé.

Après étude des algorithmes de calcul utilisés, il apparait que SOFA serait possiblement la cause de
l'instabilité. Le bug a été remonté aux developpeurs
\footnote{\url{https://wiki.sofa-framework.org/tdev/ticket/240}} pour en assurer la prise en compte.

Le but étant, bien sur, d'arriver à se passer de ce coefficient pour être le plus proche possible
du modèle développé par Jordan H. Boyle.

% subsection Modification et correction (end)

% section Implémentation du modéle (end)

\section{Locomotion arrière} % (fold)
\label{sec:Locomotion arrière}

Le circuit neuronal de la locomotion arrière est très proche du circuit neuronal de la locomotion
avant, mais inversé\cite{Boyle2009}. Il n'existe malheureusement que peu d'informations sur le circuit
permettant de passer de la locomotion avant à l'arrière, ormis le cablage du réseau.

Actuellement, l'implémentation du ver possède ce deuxième circuit générant la marche arrière. Mais
le passage de l'un à l'autre n'est pas biologiquement plausible. En s'aidant de données
éléctrophysiologiques sur certains neurones \cite{Mellem2008a,Lockery2009}, ainsi que de différentes
données sur le cablage et les interactions entre neurones
\cite{Chalfie1985,Gray2005,Chen2006,Varshney2011,Leifer2011}, nous developpons un modèle plus
plausible biologiquement pour le passage d'un circuit à l'autre.

% section Locomotion arrière (end)

\section{Pirouette} % (fold)
\label{sec:Pirouette}

La pirouette, ou omega-turn car le mouvement du ver se rapproche d'un $\Omega$, permet au ver
de réaliser un changement de direction soudain. Comparé aux autres motifs de déplacements il y a encore
moins d'information biologique sur le sujet, ormis sur le but et la fréquence de la pirouette dans
la recherche de nourriture \cite{Gray2005}.

Du fait du manque d'informations sur les neurones en jeux et les circuits neuronaux utilisés lors de
la pirouette, il est difficile de la modéliser directement. Par contre nous prévoyons d'utiliser des
vidéos de pirouette et de faire coïncider le mouvement de notre modèle sur le mouvement du ver, et
ainsi remonter de l'état musculaire à l'état neuronal, et voir quels circuits et neurones seraient actifs
lors de la pirouette.

% section Pirouette (end)

% chapter Avancement (end)
